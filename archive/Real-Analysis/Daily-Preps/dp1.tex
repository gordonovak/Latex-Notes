%document class
\documentclass[10pt,twoside]{article}

%packages
\usepackage[top=1in,bottom=0.6in,left=1in,right=1in]{geometry}
\usepackage{latexsym}
\usepackage{amssymb}
\usepackage{amsfonts}
\usepackage{amstext}
\usepackage{amsmath}
\usepackage{amsthm}
\usepackage{multicol}
\usepackage{hyperref}
\usepackage{enumerate}
\usepackage{tikz}
\usepackage{pgfplots}
\usepackage{fancyhdr}
\usepackage{xcolor, mdframed}
\pgfplotsset{
	humanaxes/.style={axis lines=center, every axis plot/.append style={very thick, mark size=3}, x axis line style=-, y axis line style=-},
	humanaxeslabels/.style={every axis x label/.style={at={(current axis.right of origin)},anchor=west},every axis y label/.style={at={(current axis.above origin)},anchor=south}},
	human/.style={humanaxes, humanaxeslabels}
}
\pgfplotsset{compat=1.16} %added beacuse of some updated tex error


%if you want to remove page numbers
%\pagestyle{empty}

%bold topics
\newcommand\topic[1]{\noindent{\bf #1}}

%definition in a box with color
\newcommand{\defn}[1]{
\begin{mdframed}[backgroundcolor=blue!05] #1
\end{mdframed}
}

%hint command
\newcommand{\hint}[1]{\noindent{\footnotesize {\it #1}}}

%theorems
\theoremstyle{plain}
\newtheorem{Theorem}{Theorem}
\newtheorem{Proposition}[Theorem]{Proposition}
\newtheorem{Corollary}[Theorem]{Corollary}
\newtheorem{Lemma}[Theorem]{Lemma}
\newtheorem{Question}[Theorem]{Question}
\newtheorem{Conjecture}[Theorem]{Conjecture}
\newtheorem{Assumption}[Theorem]{Assumption}
\newtheorem{Algorithm}[Theorem]{Algorithm}

\theoremstyle{definition}
\newtheorem{Definition}[Theorem]{Definition}
\newtheorem{Property}[Theorem]{Property}
\newtheorem{Notation}[Theorem]{Notation}
\newtheorem{Condition}[Theorem]{Condition}
\newtheorem{Example}[Theorem]{Example}
\newtheorem{Exercise}[Theorem]{Exercise}
\newtheorem{Introduction}[Theorem]{Introduction}
\theoremstyle{remark}
\newtheorem{Remark}[Theorem]{Remark}


\newcommand{\mymk}[1]{%
  \tikz[baseline=(char.base)]\node[anchor=south west, draw,rectangle, rounded corners, inner sep=2pt, minimum size=7mm,
    text height=2mm](char){\ensuremath{#1}} ;}



% here is highlighted/colored text
\newcommand{\hl}[1]{\textcolor{red}{#1}} %note that \hl{} highlights text like a highlighter
\newcommand{\hlred}[1]{\textcolor{red}{#1}}
\newcommand{\hlblue}[1]{\textcolor{blue}{#1}}
\newcommand{\hlgreen}[1]{\textcolor{green}{#1}}
\newcommand{\mathhl}[1]{\colorbox{yellow}{$#1$}}


%This will put a circle around something.
\newcommand*\circled[1]{\tikz[baseline=(char.base)]{
            \node[shape=circle,draw,inner sep=2pt] (char) {#1};}}

%These are two other examples of matrices.
%$G = \bigg\{ \begin{pmatrix} a & b \\ 0 & a \end{pmatrix} \bigg| a,b \in \mathbb{R}, a\neq 0 \bigg\}$ 
%$G = \bigg\{ \begin{bmatrix} a & b \\ 0 & a \end{bmatrix} \bigg| a,b \in \mathbb{R}, a\neq 0 \bigg\}$ 


% Commands for abstract algebra

\newcommand{\integers}{\mathbb{Z}}
\newcommand{\reals}{\mathbb{R}}
\newcommand{\complex}{\mathbb{C}}
\newcommand{\normal}{\triangleleft}
\newcommand{\rationals}{\mathbb{Q}}
\newcommand{\field}{\mathbb{F}}
\newcommand{\naturals}{\mathbb{N}}

\newcommand{\aut}[1]{{\rm Aut}(#1)}
\newcommand{\Ker}{{\rm Ker}\,}
\newcommand{\Ima}{{\rm Im}\,}
\newcommand{\cyclic}[1]{\langle #1 \rangle}
\newcommand{\isom}{\cong}

\newcommand{\NN}{\mathbb{N}}
\newcommand{\ZZ}{\mathbb{Z}}
\newcommand{\QQ}{\mathbb{Q}}
\newcommand{\RR}{\mathbb{R}}
\newcommand{\CC}{\mathbb{C}}
\newcommand{\FF}{\mathbb{F}}

%if you want to change some counters you can use the command below
%\setcounter{section}{1}

\pagestyle{fancy}
\lhead{MATH 252}
\chead{\large{\textbf{Daily Prep 01} }}
\rhead{Book section: 0.1, 0.2}
\lfoot{}
\cfoot{}
%\rfoot{\thepage/\pageref{LastPage} }
\setlength{\headheight}{14pt} %added in bc warning


\begin{document}

\newpage

\pagestyle{fancy}
\lhead{MATH 252}
\chead{\large{\textbf{Daily Prep 02}}}
\rhead{Book section: 0.2, 0.3}
\lfoot{}
\cfoot{}
%\rfoot{\thepage/\pageref{LastPage} }
\setlength{\headheight}{14pt} %added in bc warning

%\setcounter{section}{1}


\section*{Daily Prep 02}
\subsection*{More on Section 0.2}
\begin{enumerate}
 


\item Consider the sets below. $$A = \{ 0,2,4,6,8, \dots \},\   B = \{ 2k+1 \mid k \in \ZZ \}, \  C = \{0,1,2\}, \ D=\{n\in\mathbb{Z} \mid n^2<10\}$$
\begin{enumerate}
\item Find $|A|$, $|B|$, $|C|$ and $|D|$.
\begin{enumerate}
    \item $A$ is countably infinite
    \item $B$ is countably infinite
    \item $C$ is finite
    \item $D$ is finite
\end{enumerate}

\item List all the elements in $D$.
\begin{enumerate}
    \item $D=\{-3,-2,-1,0,1,2,3\}$
\end{enumerate}

\item List or describe the elements in $A \cup B$.
\begin{enumerate}
    \item $A\cup B = $ All the negative odd integers and all the natural integers.
\end{enumerate}

\item List the elements in $A\cap D$.
\begin{enumerate}
    \item $A\cap D = \{0,2\}$
\end{enumerate}

\item List the elements in $ A \setminus C$, the difference of sets $A$ and $C$.
\begin{enumerate}
    \item $A\setminus C = \{2k \mid (k + 2)\in \mathbb N\}$
\end{enumerate}

\item How many elements are in $C\times D$? Write down two of them.
\begin{enumerate}
    \item There are $21$ elements.
    \item $\{0,-3\},\;\{1,-3\}$
\end{enumerate}
\item Describe the {\bf{cartesian product}} of $A$ and $B$ in an English sentence that has no symbols except for the letters $A$ and $B$.  Write two concrete examples of elements in $A \times B$.
\begin{enumerate}
    \item The Cartesian product of $A$ and $B$ results in a set of ordered pairs for which every element of $A$ is paried with every element of $B$ in its own pair. The size of the Cartesian product is the product of the sizes of $A$ and $B$ respectively.
\end{enumerate}
\end{enumerate}
\end{enumerate}

\subsection*{Read section 0.3}

\begin{enumerate}
   \setcounter{enumi}{1}
\item At the top of page ten you will find the words {\bf domain, codomain,} and {\bf image}.  Identify these sets for the function $f:\reals\to\reals$ defined by $f(x)=x^2$.  Note:  the definition for \textbf{image} of a function as a set is near the top of page 12 just after  example 5. 
\begin{enumerate}
    \item Domain is all possible values of $x$, which is $\mathbb R$
    \item Codomain is all possible values of $x^2$ for $x$, which is $f=\mathbb R$
    \item Image is the range of possible values, which would be $[0,\infty)$
\end{enumerate}

\item \textbf{Well-definedness} is an important issue for us in this class.  Often we will describe pairings between different types of objects, but these will only be useful to us if the paring is *actually* a mapping!  Thus we will need to check if the proposed mapping is actually ``well defined.''  Consider Example 1.  Is it possible to erase an arrow from the picture depicting the mapping $\alpha$ to obtain a {\bf{well-defined}} mapping?  If so, which one(s)?
\begin{enumerate}
    \item We would remove the arrow from $2$ to $a$.
\end{enumerate}
\item Theorem 1 on page 11 tells us when two mappings are equal.  Not a big deal in calculus, but is a bigger deal in algebra.  Assuming a domain of all real numbers, decide which of these functions are equal and which are not:  $\alpha(x) = x$, $\beta(x) = |x|$, and $\gamma(x) = \sqrt{x^2}$.  Give reasons!
\begin{enumerate}
    \item $\beta$ and $\gamma$ are equal because the outputs are all positive integers, while $\alpha$ is just all integers.
\end{enumerate}


\item You may be familiar with the concepts of \textbf{one to one} \textit{(aka injective)}, \textbf{onto}\textit{(aka surjective)}, and \textbf{one to one correspondence}\textit{(aka bijective)} from prior courses (e.g., linear algebra).  These are reviewed on pages 11 and 12. 
 Using the arrow diagrams before example 4 as inspiration try to draw the following:  
 \begin{enumerate}
     
 \item A diagram depicting a function $\gamma:\{5,6,7,8\}\to\{j,k,l,m\}$ that is both one-to-one and onto. 
 \newline
 \newline
 \newline
 \newline
 \newline 
 \item A diagram depicting a function $\delta:\{5,6,7,8\}\to\{j,k,l,m\}$ that is neither one-to-one nor onto.
 \newline
 \newline
 \newline
 \newline
 \newline 

 \end{enumerate}

\item Consider Example 6 about \textbf{composite} mappings. (Note that the book writes $fg$ for the composite of $f$ and $g$, but you might have seen $f \circ g$ at some point.)  Compute both $fg(x)$ and $gf(x)$ where $f(x)=2x+1, \, g(x)=3x.$
\begin{enumerate}
    \item $f\circ g = 6x+1$
    \item $g\circ f = 6x+3$
\end{enumerate}
\item Read the \textbf{Invertibility Theorem} on p.~15 and use it to determine which  (if any) of the functions from problem 4 above is \textbf{invertible}. If one of $\gamma$ or $\delta$ is invertible, draw an arrow diagram of its inverse.
\newline
 \newline
 \newline
 \newline
 \newline \newline
 \newline
 \newline
 \newline
 \newline 

\end{enumerate}

hello


\medskip

    \hint{ \textbf{Stuck?} Ask a question in the \textbf{Forum} on Moodle.}

\end{document}
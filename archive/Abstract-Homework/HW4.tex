 %document class
\documentclass[10pt,twoside]{article}

%packages
\usepackage[top=1in,bottom=0.6in,left=1in,right=1in]{geometry}
\usepackage{latexsym}
\usepackage{amssymb}
\usepackage{amsfonts}
\usepackage{amstext}
\usepackage{amsmath}
\usepackage{amsthm}
\usepackage{multicol}
\usepackage{hyperref}
\usepackage{enumerate}
\usepackage{tikz}
\usepackage{pgfplots}
\usepackage{fancyhdr}
\usepackage{xcolor, mdframed}
\pgfplotsset{
	humanaxes/.style={axis lines=center, every axis plot/.append style={very thick, mark size=3}, x axis line style=-, y axis line style=-},
	humanaxeslabels/.style={every axis x label/.style={at={(current axis.right of origin)},anchor=west},every axis y label/.style={at={(current axis.above origin)},anchor=south}},
	human/.style={humanaxes, humanaxeslabels}
}
\pgfplotsset{compat=1.16} %added beacuse of some updated tex error


%if you want to remove page numbers
%\pagestyle{empty}


%theorems
\theoremstyle{plain}
\newtheorem{Theorem}{Theorem}
\newtheorem{Proposition}[Theorem]{Proposition}
\newtheorem{Corollary}[Theorem]{Corollary}
\newtheorem{Lemma}[Theorem]{Lemma}
\newtheorem{Question}[Theorem]{Question}
\newtheorem{Conjecture}[Theorem]{Conjecture}
\newtheorem{Assumption}[Theorem]{Assumption}
\newtheorem{Algorithm}[Theorem]{Algorithm}

\theoremstyle{definition}
\newtheorem{Definition}[Theorem]{Definition}
\newtheorem{Property}[Theorem]{Property}
\newtheorem{Notation}[Theorem]{Notation}
\newtheorem{Condition}[Theorem]{Condition}
\newtheorem{Example}[Theorem]{Example}
\newtheorem{Exercise}[Theorem]{Exercise}
\newtheorem{Introduction}[Theorem]{Introduction}
\theoremstyle{remark}
\newtheorem{Remark}[Theorem]{Remark}



%bold topics
\newcommand\topic[1]{\noindent{\bf #1}}

%definition in a box with color
\newcommand{\defn}[1]{
\begin{mdframed}[backgroundcolor=blue!05] #1
\end{mdframed}
}

%hint command
\newcommand{\hint}[1]{\noindent{\footnotesize {\it #1}}}

% here is highlighted/colored text
\newcommand{\hl}[1]{\textcolor{red}{#1}} %note that \hl{} highlights text like a highlighter
\newcommand{\hlred}[1]{\textcolor{red}{#1}}
\newcommand{\hlblue}[1]{\textcolor{blue}{#1}}
\newcommand{\hlgreen}[1]{\textcolor{green}{#1}}
\newcommand{\mathhl}[1]{\colorbox{yellow}{$#1$}}


%This will put a circle around something.
\newcommand*\circled[1]{\tikz[baseline=(char.base)]{
            \node[shape=circle,draw,inner sep=2pt] (char) {#1};}}

%These are two other examples of matrices.
%$G = \bigg\{ \begin{pmatrix} a & b \\ 0 & a \end{pmatrix} \bigg| a,b \in \mathbb{R}, a\neq 0 \bigg\}$ 
%$G = \bigg\{ \begin{bmatrix} a & b \\ 0 & a \end{bmatrix} \bigg| a,b \in \mathbb{R}, a\neq 0 \bigg\}$ 


% Commands for abstract algebra

\newcommand{\integers}{\mathbb{Z}}
\newcommand{\reals}{\mathbb{R}}
\newcommand{\complex}{\mathbb{C}}
\newcommand{\normal}{\triangleleft}
\newcommand{\rationals}{\mathbb{Q}}
\newcommand{\field}{\mathbb{F}}
\newcommand{\naturals}{\mathbb{N}}

\newcommand{\aut}[1]{{\rm Aut}(#1)}
\newcommand{\Ker}{{\rm Ker}\,}
\newcommand{\Ima}{{\rm Im}\,}
\newcommand{\cyclic}[1]{\langle #1 \rangle}
\newcommand{\isom}{\cong}

\newcommand{\NN}{\mathbb{N}}
\newcommand{\ZZ}{\mathbb{Z}}
\newcommand{\QQ}{\mathbb{Q}}
\newcommand{\RR}{\mathbb{R}}
\newcommand{\CC}{\mathbb{C}}
\newcommand{\FF}{\mathbb{F}}

\newcommand{\N}{\mathbb{N}}
\newcommand{\Z}{\mathbb{Z}}
\newcommand{\Q}{\mathbb{Q}}
\newcommand{\R}{\mathbb{R}}
\newcommand{\C}{\mathbb{C}}
\newcommand{\F}{\mathbb{F}}
\newcommand{\vp}{\vspace{0.15cm}\\}
\newcommand{\vpp}{\vspace{0.25cm}\\}
\newcommand{\vpn}{\vspace{0.05cm}\\}

%if you want to change some counters you can use the command below
%\setcounter{section}{1}

\begin{document}

\pagestyle{fancy}
\lhead{MATH 252}

\chead{\large{\textbf{Homework 03} }}
\rhead{Book section: 0.4, 1.1}
\lfoot{}
\cfoot{}
%\rfoot{\thepage/\pageref{LastPage} }
\setlength{\headheight}{14pt} %added in bc warning

\section*{Homework 04}



\subsection*{From the textbook, Section 1.2}


\begin{enumerate}

    \item Do Exercise 12. 

    \begin{enumerate}
        \item If $m$ and $n$ are relatively prime and $k|m$, show that $k$ and $n$ are relatively prime. 
        \begin{proof}
            We want to show that $\exists a, b\in \Z$ such that $ak + bn = 1$.\vp
            Suppose $k|m$. Thus, $\exists c\in\Z$ such that $k\cdot c=m$. \vp 
            Because $m$ and $n$ are relatively prime $\exists x,y\in\Z$ such that
            $$xm+yn = 1.$$
            Thus, we can rewrite this equation as:
            $$xck + yn=1.$$
            Thus, $k$ and $y$ are relatively prime.
        \end{proof}
    \end{enumerate}
    
    \item Do Exercise 16. 
    
    \begin{enumerate}
        \item If $n|k(n + 1)$, show that $n|k$.
        \begin{proof}
            We want to show that $n|k$.\vp
            Consider $n$ and $n+1$. Because they can be written as:
            $$n+1 - n =1,$$
            they are relatively prime.\vp
            Hence, via. Theorem 5, because $n|k(n+1)$, then
            $$n|k.$$
        \end{proof}
    \end{enumerate}

    \hint{Hint: $n$ and $n+1$ are coprime... then look at Theorem 5...}
   
    \item  Do Exercise 31 (a) by using Theorem 9 like in Example 8 (so using a \textit{prime decomposition}). Then check that your answer is correct using the Corollary to Theorem 9. This is an easy one, do not skip!\vpp
    Find the gcd and the lcm of the following pairs of numbers:
    \begin{enumerate}
        \item 735, 110 
        \begin{align*}
            735&=3^1 5^1 7^2\\
            110&=2^1 5^1 11^1\\
            \gcd &(735,110)=5^1=5 \\
            \text{lcm}&(735,110)=2^1 3^1 5^1 7^2 11^1=16,170
        \end{align*}
        Therefore, via. Theorem 9's Corollary, $5\cdot 16170=80850 = 735\cdot 110$.
    \end{enumerate}
\end{enumerate}



\subsection*{From the textbook, Section 1.3}

\begin{enumerate}
\setcounter{enumi}{3}

    % \item[1ac:]  Beware the negative case!
    \item Do Exercise 3 parts (a) and (b). Use carefully the definition of congruence modulo $k$!
    \begin{enumerate}
        \item $-3\equiv 7\;(\text{mod }k)$
        \begin{align*}
            -3-7=-10\\
            k=\pm(1,2,5,10)
        \end{align*}
        \item $7\equiv -5\;(\text{mod }k)$
        \begin{align*}
            7+5&=12\\
            k=\pm(1,2,\;&3,4,6,12)
        \end{align*}

    \end{enumerate}
    \item Do Exercise 7. Use carefully the definition of congruence modulo $n$!
    \begin{enumerate}
        \item If $a\equiv b\;(\text{mod } n)$ and $m|n$, show that $a\equiv b\; (\text{mod } m)$
        \begin{proof}
            We want to show that $a\equiv b\; (\text{mod } m)$.\vp
            Via the definition of congruence modulo, we know that $n\mid (a-b)$.\vp
            Additionally, we know because $m\mid n$, then we also know:
            $$m\mid (a-b).$$
            Hence, $a\equiv b\;(\text{mod }m)$
        \end{proof}
    \end{enumerate}
    
\end{enumerate}



\subsection*{From the textbook, Section 1.4}


\topic{Note:}
    A permutation is called odd if it can be written as an odd number of transposition. A permutation is called even if it can be written as an even number of transpositions. The parity of a permutation is its evenness/oddness property. It is a theorem that, even though a permutation can be written in many ways as a product of transpositions, parity is a well-defined property!

\begin{enumerate}
 \setcounter{enumi}{5}   
    \item Do Exercise 13 parts parts (a) and (b).\\
    Factor each of the following equation into disjoin cycles, find its parity, and factor the inverse into disjoint cycles.
    \begin{enumerate}
        \item \begin{equation*}
                \sigma=\left(\begin{array}{@{}*{20}{c@{}}}
                  1 \;& 2\; & 3 \;& 4 \;& 5 \;& 6 \;& 7 \;& 8 \;&9\\ 
                  4 \;& 7\; & 9 \;& 8 \;& 2 \;& 1 \;& 6 \;& 3 \;&5
                \end{array}\right)
            \end{equation*}
            $$\sigma = (1,4,8,3,9,5,2,7,6)$$
            $$\sigma^{-1} = (1,6,7,2,5,9,3,8,4)$$
            This means that our parity is $9-1=8$.
        \item \begin{equation*}
            \sigma=\left(\begin{array}{@{}*{20}{c@{}}}
              1 \;& 2\; & 3 \;& 4 \;& 5 \;& 6 \;& 7 \;& 8 \;&9\\ 
              3 \;& 8\; & 9 \;& 5 \;& 2 \;& 1 \;& 6 \;& 4 \;&7
            \end{array}\right)
        \end{equation*}
        $$\sigma = (1,3,9,7,6)(2,8,4,5)$$
        $$\sigma^{-1} = (1,6,7,9,3)(2,5,4,8)$$
        This means that our parity is $(5-1)+(4-3)=7$.
    \end{enumerate}

\hint{Hint: Make sure to look at the theorem in the book that tells you the parity of a cycle of length $k$.}\\

    \item Do Exercise 20. 
    \begin{enumerate}
        \item Show that $(1,2)$ is not a product of 3-cycles.
        \begin{proof}
            Let $\sigma = (a,b,c) \; \& \; \epsilon = (d,e,f)$ be cycles.\\
            We want to show that $\sigma \cdot \epsilon$ will always result in an even parity.\\
            $$\text{Parity }\sigma \cdot \epsilon=(3-1)+(3-1) = 4$$
            Becuase the result of 3-cycle multiplication always results in an even parity, 
            it can \textit{never} be represented as an odd number of transpositions.\vp
            Hence, because $(1,2)$ is an odd number of transpositions, it is not a product of 3-cycles.
        \end{proof}
    \end{enumerate}
    
    \hint{Hint: parity.}
\end{enumerate}

\subsection*{Extra Credit}
\begin{enumerate}
 \setcounter{enumi}{7}  
    \item  Section 1.2,  Exercise 18. \vp
    If $\gcd(m, n) = 1$, let $d = \gcd(m + n, m- n)$. Show that $d = 1$ or $d = 2$.
    \begin{proof}\textit{Direct. }\vp
        Because $\gcd(m,n)=1$, we know that $\exists a,b\in\R$ such that $am+bn = 1$.\\
        Additionally, because $\gcd(m+n,m-n)=d$, it reasons that:
        \begin{align*}
            d&\mid m+n \text{ (1)}\\
            d&\mid n-m\text{ (2)}\\
            fd&=m+n\\
            gd&=m-n\\
            (f+g)d&=2m\\
            (f-g)d&=2n\\
            \text{Therefore, } &d \text{ divides } 2m \;\&\;2n,\\
        \end{align*}
        And because $\gcd$ of $m,n$ is 1, we have that $d$ is either $1$ or $2$.
    \end{proof}
    
    \hint{Recall that if $d \mid x$ and $d \mid y$, then $d \mid ax+by$ for any choice of $a,b \in \ZZ$. In particular, $d \mid x+y$ and $d \mid x-y$...}

    \item Section 1.3, Exercise 12.\\
    \begin{enumerate}
        \item We want to show that each case of $a^2\mod 4$ results in $\bar 1$, $\bar 0$.\\
        If $a=\bar 0$, $a^2 = \bar 0$. \\
        If $a=\bar 1$, $a^2 = \bar 1$. \\
        If $a=\bar 2$, $a^2 = \bar 4 = \bar 0$. \\
        If $a=\bar 3$, $a^2 = \bar 9 = \bar 1$.\\
        Therefore, $a^2$ in mod 4 is either $\bar 0$ or $\bar 1$. 
        \item We want to show that no number following the pattern is a perfect square: 11111...\\
        Because $100\mod 4$ is $\bar 0$, and every $10^n\cdot 100$ is a multiple of 100, they will all be $\bar 0$.\vp
        Therefore, only 1 and 11 contribute to the modulus, and $11\mod 4 = \bar 3$, and no perfect square can be $\bar 3$ as shown above.\vp
        Hence no number following this pattern can be a perfect square. 
    \end{enumerate}
     \hint{Hint for part (b): 111 = 100 + 10 + 1 using the decimal expansion of a number. Then you can consider each summand mod 4...}


 
\item (Not from the book on the content from Section 1.4). You can write the product of any two transpositions $(a \,  b)(c \,  d)$ as a product of 3-cycles! Show this is possible by considering three cases: 
\begin{enumerate}[(i)]
    \item $a, b, c, d$ are all different numbers;\\
        $(c\;b\;d)(a\;c\;b)=(a\;b)(c\;d)$
    \item  $(a \, b) = (c \, d)$; \hint{What happens if you compose the same 3-cycle 3 times?}\\
        $(a\;b\;c)^3=\varepsilon=(a\;b)(c\;d)$, if $(a\;b)=(c\;d)$.
    \item there's a common number such as $b=c$, with $a \neq d$.\\
        $(a\;b\;d)^4=\varepsilon\cdot (a\;b\;d)=(a\;b)(c\;d)$
\end{enumerate}

\end{enumerate}

\end{document}
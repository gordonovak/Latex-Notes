%document class
\documentclass[10pt,twoside]{article}

%{
%packages
\usepackage[top=1in,bottom=0.6in,left=1in,right=1in]{geometry}
\usepackage{latexsym}
\usepackage{amssymb}
\usepackage{amsfonts}
\usepackage{amstext}
\usepackage{amsmath}
\usepackage{amsthm}
\usepackage{multicol}
\usepackage{hyperref}
\usepackage{enumerate}
\usepackage{tikz}
\usepackage{pgfplots}
\usepackage{array}
\usepackage{fancyhdr}
\usepackage{xcolor, mdframed}
\usepackage{enumitem}
\pgfplotsset{
    humanaxes/.style={axis lines=center, every axis plot/.append style={very thick, mark size=3}, x axis line style=-, y axis line style=-},
    humanaxeslabels/.style={every axis x label/.style={at={(current axis.right of origin)},anchor=west},every axis y label/.style={at={(current axis.above origin)},anchor=south}},
    human/.style={humanaxes, humanaxeslabels}
}
\pgfplotsset{compat=1.16} %added beacuse of some updated tex error


%if you want to remove page numbers
%\pagestyle{empty}


%theorems
\theoremstyle{plain}
\newtheorem{Theorem}{Theorem}
\newtheorem{Proposition}[Theorem]{Proposition}
\newtheorem{Corollary}[Theorem]{Corollary}
\newtheorem{Lemma}[Theorem]{Lemma}
\newtheorem{Question}[Theorem]{Question}
\newtheorem{Conjecture}[Theorem]{Conjecture}
\newtheorem{Assumption}[Theorem]{Assumption}
\newtheorem{Algorithm}[Theorem]{Algorithm}

\theoremstyle{definition}
\newtheorem{Definition}[Theorem]{Definition}
\newtheorem{Property}[Theorem]{Property}
\newtheorem{Notation}[Theorem]{Notation}
\newtheorem{Condition}[Theorem]{Condition}
\newtheorem{Example}[Theorem]{Example}
\newtheorem{Exercise}[Theorem]{Exercise}
\newtheorem{Introduction}[Theorem]{Introduction}
\theoremstyle{remark}
\newtheorem{Remark}[Theorem]{Remark}



%bold topics
\newcommand\topic[1]{\noindent{\bf #1}}

%definition in a box with color
\newcommand{\defn}[1]{
\begin{mdframed}[backgroundcolor=blue!05] #1
\end{mdframed}
}

%hint command
\newcommand{\hint}[1]{\noindent{\footnotesize {\it #1}}}

% here is highlighted/colored text
\newcommand{\hl}[1]{\textcolor{red}{#1}} %note that \hl{} highlights text like a highlighter
\newcommand{\hlred}[1]{\textcolor{red}{#1}}
\newcommand{\hlblue}[1]{\textcolor{blue}{#1}}
\newcommand{\hlgreen}[1]{\textcolor{green}{#1}}
\newcommand{\mathhl}[1]{\colorbox{yellow}{$#1$}}


%This will put a circle around something.
\newcommand*\circled[1]{\tikz[baseline=(char.base)]{
            \node[shape=circle,draw,inner sep=2pt] (char) {#1};}}

%These are two other examples of matrices.
%$G = \bigg\{ \begin{pmatrix} a & b \\ 0 & a \end{pmatrix} \bigg| a,b \in \mathbb{R}, a\neq 0 \bigg\}$ 
%$G = \bigg\{ \begin{bmatrix} a & b \\ 0 & a \end{bmatrix} \bigg| a,b \in \mathbb{R}, a\neq 0 \bigg\}$ 


% Commands for abstract algebra

\newcommand{\integers}{\mathbb{Z}}
\newcommand{\reals}{\mathbb{R}}
\newcommand{\complex}{\mathbb{C}}
\newcommand{\normal}{\triangleleft}
\newcommand{\rationals}{\mathbb{Q}}
\newcommand{\field}{\mathbb{F}}
\newcommand{\naturals}{\mathbb{N}}

\newcommand{\aut}[1]{{\rm Aut}(#1)}
\newcommand{\Ker}{{\rm Ker}\,}
\newcommand{\Ima}{{\rm Im}\,}
\newcommand{\cyclic}[1]{\langle #1 \rangle}
\newcommand{\isom}{\cong}

\newcommand{\NN}{\mathbb{N}}
\newcommand{\ZZ}{\mathbb{Z}}
\newcommand{\QQ}{\mathbb{Q}}
\newcommand{\RR}{\mathbb{R}}
\newcommand{\CC}{\mathbb{C}}
\newcommand{\FF}{\mathbb{F}}

\newcommand{\N}{\mathbb{N}}
\newcommand{\Z}{\mathbb{Z}}
\newcommand{\Q}{\mathbb{Q}}
\newcommand{\R}{\mathbb{R}}
\newcommand{\C}{\mathbb{C}}
\newcommand{\F}{\mathbb{F}}

% my commands
\newcommand{\vp}{\vspace{0.15cm}\\}
\newcommand{\vpp}{\vspace{0.25cm}\\}
\newcommand{\vpn}{\vspace{0.05cm}\\}
\newcommand{\twom}[4]{\begin{bmatrix}#1 & #2 \\ #3 & #4\end{bmatrix}}
\newcommand{\rmv}[1]{\,\backslash\{#1\}}
\newcommand{\casi}[4]{\begin{cases}#1 & \text{ if }#2\\ #3 & \text{ if }#4\end{cases}}
\usepackage{changepage}
\usepackage{array}

%if you want to change some counters you can use the command below
%\setcounter{section}{1}

\begin{document}

\section*{Homework 11}

\pagestyle{fancy}
\lhead{MATH 252}
\chead{\large{\textbf{Homework 11}}}
\rhead{Book section: 4.1, 4.2}
\lfoot{}
\cfoot{}
%\rfoot{\thepage/\pageref{LastPage} }
\setlength{\headheight}{14pt} %added in bc warning

%\setcounter{section}{1}


\subsection*{From the textbook, Section 4.1}
\begin{enumerate}
    %%%%%%%%%%%%%%%%%%%
    %%%%%%%% Exercise 3
    \item \underline{Do Exercise 3. Make sure it is clear how you deduced your answers!}
    \begin{enumerate}
        %%%% Part a
        \item How many polynomials of degree 3 are there in $\Z_5[x]?$.\vp
        Consider the possibilities of each term in the polynomial:
        \begin{align*}
            x^3&: x^3, \;2x^3, \;3x^3, \;4x^3\\
            x^2&: 0, \;x^2,\;2x^2,\;3x^2,\;4x^2\\
            x&: 0,\;x,\;2x,\;3x,\;4x\\
            1&: 0,\; 1,\; 2,\; 3,\; 4
        \end{align*}
        This gives a total number of combinations of $4\cdot 5\cdot 5\cdot 5 = 500$. 
        
        %%%% Par b
        \item How many monic polynomials of degree 3 are there in $\Z_3[x]$?\vp
        Consider the possibilities again:
        \begin{align*}
            x^3&: x^3\\
            x^2&: 0, \;x^2,\;2x^2\\
            x&: 0,\;x,\;2x\\
            1&: 0,\; 1,\; 2
        \end{align*}
        This gives us a total number of combinatinos of $1\cdot 3\cdot 3\cdot 3 = 27$.

    \end{enumerate}
    %%%%%%%%%%%%%%%%%%%
    %%%%%%%% Exercise 16
    \item \underline{Do Exercise 16 (a). Hint: The Factor/Remainder Theorem might help!}\vp
    For which primes $p$ is $x-1$ a factor of $f = 3x^4+5x^3+2x^2+x+4$ in $\Z_p$?
    \begin{proof}
        We want to show that for $p = 3,5$, $(x-1)$ is a factor of $f = 3x^4+5x^3+2x^2+x+4$ in $\Z_p[x]$?.\vp
        We will not consider cases for $p>6$, because then the polynomial doesn't change.\vp
        Consider the special case where $f(1)=0$ in $\Z_p$.\\
        Listing out the polynomials with their results at $1$ shows this clearly:
        \begin{align*}
            \Z_5 : \;&f = 3x^4 + 2x^2 + x + 4\\
            &f(1) = 15\equiv 0 \text{mod } 5\\
            \Z_3 : \;&f = 3x^{4}+1x^{3}+2x^{2}+x\\
            &f(1)=3\equiv 0 \text{mod } 3\\
            \Z_2 :\; &f = 2x^{3}+2x^{2}+x+1\\
            &f(1)= 3 \equiv 1 \text{mod }2
        \end{align*}
        Hence, $(x-1)$ is a factor of $f$ in $\Z_3[x]$ and $\Z_5[x]$. 
    \end{proof}

    %%%%%%%%%%%%%%%%%%%
    %%%%%%%% Exercise 17
    \item \underline{Do Exercise 17 (b). Hint: The Factor/Remainder Theorem might help!}\vp
    In each case factor $f$ into linear factors in $F[x]$. 
    \begin{enumerate}
        \item[(b)]$f = x^3 + 1, \;F = \Z_7$. \vp
        All I'm going to do is search for $x^3 \equiv 6 \text{mod }7$ because then I can add the $1$ to get to a root. For $g=x^3\text{mod }7$:
        \begin{align*}
            g(0)&=0\\
            g(1)&=1\\
            g(2)&=8=2\\
            g(3)&=27=6 \text{ (this one)}\\
            g(4)&=64=1\\
            g(5)&=125=6\text{ (this one)}\\
            g(6)&=216=6\text{ (this one)}
        \end{align*}
        So we factor as:
        \begin{align*}
            f &= (x-3)(x-5)(x-6)\\
            &=(x+4)(x+2)(x+1)
        \end{align*}
    \end{enumerate}
\end{enumerate}



\subsection*{From the textbook, Section 4.2}
\begin{enumerate}
% problems 10, 19, 21a, 22b, Hints: Use the Modular Irreducibility Test and Eisenstein's Criterion.
\setcounter{enumi}{3}
    \item \underline{Do Exercise 10.}\vp
    We are going to find all irreducible cubic, primarily by finding if they have a root or not. 
    \begin{align*}
        f_1(x)&=x^3+x^2+x+1\\
        f_2(x)&=x^3+x^2+1\\
        f_3(x)&=x^3+x^2+x\\
        f_4(x)&=x^3+x^2\\
        f_5(x)&=x^3+x+1\\
        f_6(x)&=x^3+1\\
        f_7(x)&=x^3+x\\
        f_8(x)&=x^3
    \end{align*}
    We immediately know $f_3, f_4, f_8$ are reducible because they all have $x$ terms. We also rule out $f_6$ and $f_1$ because $f_6(1)=0, f_1(1)=0$. \vp
    Then, consider that $f_2(0)=1, f_2(1)=1$. Thus, $f_2$ is irreducible. The same applies to $f_5$. 
    
    \item \underline{Do Exercise 19. You will need an irreducible cubic from Exercise 10!}\vp
    Factor $x^5+x^4+1$ as a product of irreducible polynomials in $\Z_2$. \vp
    If we factor out $f_5(x)$ from from the previous problem, we get from the divison algorithm that:
    \begin{align*}
        f(x)\div f_5(x)= x^2+x
    \end{align*}
    With no remainder! So we can factor our function as:
    \begin{align*}
        f(x) = (x^2+x)(x^3+x+1)
    \end{align*}
    \item \underline{Do Exercise 21 (a). Hint: Use the Modular Irreducibility Test or Eisenstein's Criterion}\vp
    Show that each polynomial is irreducible in $\Q[x]$. 
    \begin{enumerate}
        \item $3x^3+5x^2+x+2$\vp
        We will use the Modular Irreducibility Test. Let $p=5$. In $\Z_5$ we have:
        \begin{align*}
            f(x) = 3x^3+x+2
        \end{align*}
        Consider that for even $x$, $3x^3+x$ is a nonzero even. For odd $x$, $3x^3+x$ is also a nonzero even. Thus, because $f(x)$ is always even and nonzero in $\Z_5$, it will never equal a multiple of 5, so it is irreducible. \vp
        Thus, via the Modular Irreducibility Test $3x^3+5x^2+x+2$ is irreducible in $\Q[x]$. 

    \end{enumerate}
    \item \underline{Do Exercise 22 (b).  Hint: Use the Modular Irreducibility Test or Eisenstein's Criterion}\vp
    Show that each polynomial is irreducible in $\Q[x]$. 
    \begin{enumerate}
        \item [(b)] $4x^5+28x^4+7x^3-28x^2+14$\vp
        We will use Eisenstein's Criterion. Consider $p=7$. \\
        $p$ divides everything except the leading coefficient because they are all multiples of 7\vp
        Additionally, $p^2=49$ does not divide $a_0=14$. \vp
        Thus, via Eisenstein's Criterion, $4x^5+28x^4+7x^3-28x^2+14$ is irreducible in $\Q[x]$. 
    \end{enumerate}
\end{enumerate}


\end{document}
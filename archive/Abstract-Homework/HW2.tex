%document class
\documentclass[10pt,twoside]{article}

%packages
\usepackage[top=1in,bottom=0.6in,left=1in,right=1in]{geometry}
\usepackage{latexsym}
\usepackage{amssymb}
\usepackage{amsfonts}
\usepackage{amstext}
\usepackage{amsmath}
\usepackage{amsthm}
\usepackage{multicol}
\usepackage{hyperref}
\usepackage{enumerate}
\usepackage{tikz}
\usepackage{pgfplots}
\usepackage{fancyhdr}
\usepackage{xcolor, mdframed}
\pgfplotsset{
	humanaxes/.style={axis lines=center, every axis plot/.append style={very thick, mark size=3}, x axis line style=-, y axis line style=-},
	humanaxeslabels/.style={every axis x label/.style={at={(current axis.right of origin)},anchor=west},every axis y label/.style={at={(current axis.above origin)},anchor=south}},
	human/.style={humanaxes, humanaxeslabels}
}
\pgfplotsset{compat=1.16} %added beacuse of some updated tex error


%if you want to remove page numbers
%\pagestyle{empty}


%theorems
\theoremstyle{plain}
\newtheorem{Theorem}{Theorem}
\newtheorem{Proposition}[Theorem]{Proposition}
\newtheorem{Corollary}[Theorem]{Corollary}
\newtheorem{Lemma}[Theorem]{Lemma}
\newtheorem{Question}[Theorem]{Question}
\newtheorem{Conjecture}[Theorem]{Conjecture}
\newtheorem{Assumption}[Theorem]{Assumption}
\newtheorem{Algorithm}[Theorem]{Algorithm}

\theoremstyle{definition}
\newtheorem{Definition}[Theorem]{Definition}
\newtheorem{Property}[Theorem]{Property}
\newtheorem{Notation}[Theorem]{Notation}
\newtheorem{Condition}[Theorem]{Condition}
\newtheorem{Example}[Theorem]{Example}
\newtheorem{Exercise}[Theorem]{Exercise}
\newtheorem{Introduction}[Theorem]{Introduction}
\theoremstyle{remark}
\newtheorem{Remark}[Theorem]{Remark}



%bold topics
\newcommand\topic[1]{\noindent{\bf #1}}

%definition in a box with color
\newcommand{\defn}[1]{
\begin{mdframed}[backgroundcolor=blue!05] #1
\end{mdframed}
}

%hint command
\newcommand{\hint}[1]{\noindent{\footnotesize {\it #1}}}

% here is highlighted/colored text
\newcommand{\hl}[1]{\textcolor{red}{#1}} %note that \hl{} highlights text like a highlighter
\newcommand{\hlred}[1]{\textcolor{red}{#1}}
\newcommand{\hlblue}[1]{\textcolor{blue}{#1}}
\newcommand{\hlgreen}[1]{\textcolor{green}{#1}}
\newcommand{\mathhl}[1]{\colorbox{yellow}{$#1$}}


%This will put a circle around something.
\newcommand*\circled[1]{\tikz[baseline=(char.base)]{
            \node[shape=circle,draw,inner sep=2pt] (char) {#1};}}

%These are two other examples of matrices.
%$G = \bigg\{ \begin{pmatrix} a & b \\ 0 & a \end{pmatrix} \bigg| a,b \in \mathbb{R}, a\neq 0 \bigg\}$ 
%$G = \bigg\{ \begin{bmatrix} a & b \\ 0 & a \end{bmatrix} \bigg| a,b \in \mathbb{R}, a\neq 0 \bigg\}$ 


% Commands for abstract algebra

\newcommand{\integers}{\mathbb{Z}}
\newcommand{\reals}{\mathbb{R}}
\newcommand{\complex}{\mathbb{C}}
\newcommand{\normal}{\triangleleft}
\newcommand{\rationals}{\mathbb{Q}}
\newcommand{\field}{\mathbb{F}}
\newcommand{\naturals}{\mathbb{N}}

\newcommand{\aut}[1]{{\rm Aut}(#1)}
\newcommand{\Ker}{{\rm Ker}\,}
\newcommand{\Ima}{{\rm Im}\,}
\newcommand{\cyclic}[1]{\langle #1 \rangle}
\newcommand{\isom}{\cong}

\newcommand{\NN}{\mathbb{N}}
\newcommand{\ZZ}{\mathbb{Z}}
\newcommand{\QQ}{\mathbb{Q}}
\newcommand{\RR}{\mathbb{R}}
\newcommand{\CC}{\mathbb{C}}
\newcommand{\FF}{\mathbb{F}}
\newcommand{\N}{\mathbb{N}}
\newcommand{\Z}{\mathbb{Z}}
\newcommand{\Q}{\mathbb{Q}}
\newcommand{\R}{\mathbb{R}}
\newcommand{\C}{\mathbb{C}}
\newcommand{\F}{\mathbb{F}}
\newcommand{\vp}{\vspace{0.15cm}\\}
\newcommand{\vpp}{\vspace{0.25cm}\\}
\newcommand{\vpn}{\vspace{0.05cm}\\}
\newcommand{\br}[1]{\{ #1 \}}
\newcommand{\brs}[2]{\{ #1 \;|\; #2 \}}

%if you want to change some counters you can use the command below
%\setcounter{section}{1}

\pagestyle{fancy}
\lhead{MATH 252}
\chead{\large{\textbf{Homework 01} }}
\rhead{Book section: 0.1}
\lfoot{}
\cfoot{}
%\rfoot{\thepage/\pageref{LastPage} }
\setlength{\headheight}{14pt} %added in bc warning


\begin{document}



\subsection*{Instructions}
\section*{Homework 02}
\subsection*{From the textbook, Section 0.2}

\begin{enumerate}
%\setcounter{enumi}{3}
 \item Do Exercise 1 and Exercise 2(a),(b),(c),(d) as written.
 \begin{enumerate}
  \item \begin{enumerate}
      \item A = $\{x \;|\; n\in\naturals, x=5n\}$
      \item A = $\{x\in\integers\;| -\frac{1}2<x<\frac{9}2\}$
    \end{enumerate}
  \item \begin{enumerate}
    \item $\{1,3,5,...\}$ (Add odd positive integers)
    \item $\{1,2,3,4,5,6,7\}$ (All naturals less than 8)
    \item $\{-3,-1,1\}$
    \item $\{1,\frac{1}4,\frac{1}{16},\frac{1}{27},...\}$
  \end{enumerate}
  

 \end{enumerate}
 
\item Do Exercise 3 as written.
 \begin{enumerate}
  \item $A=\{n\in\integers\;|\;n^2\le4\}\ne\{x\in\reals\;|\;x^2-3x+2\}$
  \begin{itemize}
    \item These are not equal because the first contains negative integers while the second does not.
  \end{itemize}
  \item $A=\{n\in\integers\;|\;n=\frac{1}n\}=\{x\in\reals\ \;|\;x^2=1\}$
  \begin{itemize}
    \item These sets are equal because they contain only $\{1\}$
  \end{itemize}
  \item The set of letters in "alloy" = The set of letters in "loyal"
  \begin{itemize}
    \item These are equal because loyal and alloy contain the same letters
  \end{itemize}
  \item $A=\{2,\{3\},4\}\ne\{2,\{3,4\}\}$
  \begin{itemize}
    \item These are \textit{not} because $\{3,4\}$ is its own set distinct from $\{\{3\},4\}$.
  \end{itemize}
  \item $A=\{1\}\ne\{\{1\}\}$
  \begin{itemize}
    \item These are \textit{not} equal. The set containing 1 is not equal to 1 itself.
  \end{itemize}
  \item $A=\{x\in\reals\;|\;x^2=-1\}=\{x\in\rationals\;|\;x^2=2\}$
  \begin{itemize}
    \item These sets are equal because they are both the empty set ($\emptyset$)
  \end{itemize}
  \item $A=\{x\in\integers\;|\;x^2\le1\}=\{x\in\reals\;|\; x^3=x\}$
  \begin{itemize}
    \item Both sets are equal to $\{1,0,-1\}$
  \end{itemize}
 \end{enumerate}
\item Do Exercise 10 as written.  
  \topic{Hints for Exercise 10}: (a) To show that two sets are equal, use the Principle of Set Equality. (b) Look back at the worksheet from Day 01 for the structure of a biconditional proof and the structure of a proof with an ``or" statement.).
  \begin{enumerate}
    \item \textit{Proof:} Suppose $A\times B= B\times A$. Let $a\in A$ and $b\in B$. $A\ne\emptyset, B\ne\emptyset$.\\
    We want to show that for sets $A$ and $B$, $A\subseteq B$, and $B\subseteq A$.\\
    $$\text{Case 1: }A\subseteq B$$
    Because $A\times B= B\times A$, if $\{a,b\}\in A\times B$, then $\{a,b\}\in B\times A$.\\
    Therefore, $\forall a\in A,a\in B$. Hence, $A\subseteq B$.
    $$\text{Case 2: }B\subseteq A$$
    Because $B\times A= A\times B$, if $\{b,a\}\in B\times A$, then $\{b,a\}\in A\times B$.\\
    Therefore, $\forall b\in B,b\in A$. Hence, $B\subseteq A$.\\ \\
    Hence, because $A\subseteq B$, and $B\subseteq A$, $A=B$.  $\square$'

    \item \textit{Proof:} Suppose that for sets $A$ and $B$, $A=B$ or $A=\emptyset$.\\
    We want to show that for either case,  $A\times B= B\times A$.
    $$\text{Case 1: }A = B$$
    Let $x\in A$ and $y\in B$.\\
    Therefore, $\{x, y\} \in A\times B$\\
    If $A=B$, then $x\in B$, and $y\in A.$ \\
    Therefore $\{x, y\}\in B\times A$.
    Because $\forall x, \forall y, \{x,y\}\in A\times B \text{ and }B\times A,$\\
    $A\times B = B\times A$.
    $$\text{Case 2: }A = \emptyset$$
    If $A=\emptyset$\\
    $A\times B = \emptyset.$\\
    $B\times A = \emptyset.$\\
    Therefore, $A\times B = B\times A.\;\;\square$ 

    \item \begin{proof}
      Let $A$ and $B$ be sets.\\
      We want to show that:
      \begin{enumerate}
        \item $A\times B\subseteq \{x\;|\; (x,x)\in A\times B\}$\\
        and:
        \item $A\times B\supseteq \{x\;|\; (x,x)\in A\times B\}$\vspace{0.1cm}
      \end{enumerate}
      \underline{$A\cap B\subseteq \{x\;|\; (x,x)\in A\times B\}$}\vp
      We want to show that $\forall x\in A\cap B, x\in\brs{x}{(x,x)\in A\times B}$.\\
      $x\in A\cap B$ if $x\in A$ and $x\in B$. \\
      Therefore, $(x,x)\in A\times B$, and thus $x$ is in $\brs{x}{(x,x)\in A\times B}$.
      Hence, $A\cap B\subseteq \{x\;|\; (x,x)\in A\times B\}$\vp
      \underline{$A\cap B\supseteq \{x\;|\; (x,x)\in A\times B\}$}\vp
      We want to show that $\forall x\in\brs{x}{(x,x)\in A\times B}, x\in A\cap B, $.\\
      If $x\in \brs{x}{(x,x)\in A\times B}$, then $(x,x)\in A\times B$.\\
      Thus, $x\in A$ and $x\in B$. Therefore, $x\in A\cap B$. \\
      Hence, $A\cap B\supseteq \{x\;|\; (x,x)\in A\times B\}$.\vpp
      Hence $A\cap B = \{x\;|\; (x,x)\in A\times B\}$.

      
    \end{proof}

    
  \end{enumerate}
  
\end{enumerate}

\subsection*{From the textbook, Section 0.3}


\begin{enumerate}
\setcounter{enumi}{3}
    
    \item Look at the ``maps'' from Exercise 1. Find the ones that are  *not* well-defined and justify your answer.
    \begin{enumerate}
      \item Not well defined. Goes outside the natrual numbers
      \item Well defined. Codomain is entirely within $\N$. 
      \item Not well defined. Negative numbers do not work because they produce complex outputs.
      \item Well defined. A $\R\times\R$ input always goes to a $\R$ output.
      \item Not well defined. One input can result in multiple possible outputs. 
      \item Not well defined, because $2$ goes to both $b$ and $c$.
      \item Not well defined, because there's no defined behavior for $2$.
    \end{enumerate}
    \item Do Exercise 3 (c) and (d) as written.
   \topic{Hints for Exercise 3}: (c) To show a map is injective, assume that two images are the same, then show that the inputs have to be the same. (d) Is is enough to take any element $b \in B$ and somehow find some element $a\in A$ such that $\alpha(a)=b$.
   \begin{enumerate}
    \item \textit{Proof:} Suppose $\beta\alpha$ is one-to-one, then $\beta\alpha(x_1)\ne\beta\alpha(x_2)$ if $x_1\ne x_2$, \\
          If $\alpha$ is onto, then $\forall b\in B, \exists a\in A$ s.t. $\alpha(a)=b$\\
          Therefore, because $\alpha(a)=b$, $\beta(b_1)\ne\beta(b_2)$ for $b_1 \ne b_2$.\\
          Hence, $\beta$ is one-to-one. $\square$
    \item \textit{Proof:} Suppose $\beta\alpha$ is onto, then $\forall c\in C, \exists a\in A$ s.t. $\beta\alpha(a)=c$.\\
          Suppose $\beta$ is one-to-one. Then, $\exists c\in C, \exists b_1,b_2\in B$ such that $\beta(b_1)\ne\beta(c_2)$.\\
          Therefore, because each $\beta$ input ($b$) is linked directly to one $\beta$ output ($c$),\\
          If $\beta\alpha(a)=c$, $\exists\text{ unique } b\in B$ s.t. $\beta(b)=c$. \\
          Therefore, $\forall b\in B, \exists a\in A$ such that $\alpha(a)=b$ because $\beta$ is one-to-one.\\  
          Hence, $\alpha$ is onto. $\;\square$.

  \end{enumerate}

  \end{enumerate}



\end{document}
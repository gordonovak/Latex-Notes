\documentclass{article}
\usepackage{graphicx} % Required for inserting images

% Packages to make things pretty
\usepackage[utf8]{inputenc}
\usepackage{hyperref}
\usepackage{amsmath}
\usepackage{amssymb}
\usepackage{amsthm}
\usepackage{xcolor}
\usepackage{enumitem}
\usepackage[mathscr]{euscript}
\usepackage{verbatim}
\usepackage{graphicx}
\usepackage[margin=1in]{geometry}
\usepackage{multicol}
\usepackage{mathrsfs} %for right font for power set P
\everymath{\displaystyle} %makes all math display style (bigger and easier to read)
\usepackage[normalem]{ulem}
\usepackage{remreset}
\usepackage{titlesec}

\makeatletter
  \@removefromreset{subsection}{section}
\makeatother
%\renewcommand{\thesection}{}
\titleformat{\section}
{\normalfont\Large\bfseries}{}{0pt}{}
\renewcommand{\thesubsection}{Problem \arabic{subsection}}
%\newcommand{\HW}[1]{\section*{#1}}
\titleformat{\subsection}
{\normalfont\large\bfseries}{}{0pt}{}

% Theorem Environments
\theoremstyle{definition}
\newtheorem{problem}{Problem}[section]
\newtheorem{conjecture}[section]{Conjecture}
\newtheorem*{definition}{Definition}
\newtheorem*{claim}{Claim}
\newtheorem{theorem}{Theorem}[section]
\newcounter{tmp}
\newtheorem{question}[theorem]{Question}
\newtheorem*{answer}{Answer}
\newtheorem{challenge}[section]{Challenge}
\newtheorem*{postulate}{Postulate}

% These are a bunch of my own personal macros, i.e. command shortcuts I've come up with for common things I use frequently. Feel free to use or ignore them!
\newcommand{\lcu}{\left\lbrace}
\newcommand{\rcu}{\right\rbrace}
\newcommand{\lbr}{\left[}
\newcommand{\rbr}{\right]}
\newcommand{\lpa}{\left(}
\newcommand{\rpa}{\right)}
\newcommand{\lan}{\left\langle}
\newcommand{\ran}{\right\rangle}
\newcommand{\bfi}{\mathbf{i}}
\newcommand{\bfj}{\mathbf{j}}
\newcommand{\bfk}{\mathbf{k}}
\newcommand{\R}{\mathbb{R}}
\newcommand{\N}{\mathbb{N}}
\newcommand{\Z}{\mathbb{Z}}
\newcommand{\Q}{\mathbb{Q}}
\newcommand{\C}{\mathbb{C}}
\newcommand{\F}{\mathbb{F}}
\newcommand{\T}{\mathbb{T}}
\newcommand{\Epsilon}{\mathcal{E}}
\newcommand{\blue}[1]{\textcolor{blue}{#1}}
\newcommand{\FGR}{{\bf First attempt grade: R}}
\newcommand{\SGR}{{\bf Second attempt grade: R}}
\newcommand{\TGR}{{\bf Third attempt grade: R}}
\newcommand{\BGR}{{\bf Bonus attempt grade: R}}
\newcommand{\FR}{{\bf Final grade:R}}
\newcommand{\FGM}{{\bf Final grade: M}}
\newcommand{\nl}{{\bf Final grade: M}}






\title{\textbf{Math 244 HW1 Novak}}
\author{Gordon Novak}
\date{\today}

%%*****NTS: Make the problems subsections for easier scrolling.
%%*****NTS: Carry forward the section/subsection pattern.



% Actual start of the document
\begin{document}

\maketitle

\section*{HW1: First attempt due \textbf{Wednesday 2/12}}
{\it Note: For this first homework assignment, I am not yet holding you to any rigorous proof-writing standards. These will come later.}
\subsection*{Problem 1} A subset $A$ of $\R$ is called a {\bf $G_\delta$-set} ({\it pronounced ``G-delta-set"}) if $A$ can be written as a countable intersection of open intervals. In other words, $A$ is a $G_\delta$-set if there exist countably-many open intervals $U_n$ such that
	\[A = \bigcap_{n=1}^\infty U_n. \]
Notice that this includes intersections of finitely many open intervals.
	\begin{enumerate}[label=(\alph*)]
	\item Show that $(3,10)$ is a $G_\delta$-set.\\
    
    		$Proof:$ Let $A = (3,10)$. WTS $A$ can be written as $A =\bigcap_{n=1}^\infty U_n. $\\
            \begin{gather*}
                \text{Let } U_n = (4-n, \; 9+n) \\
                \text{Let }k\in \N, \text{ s.t. } k>n. \\
                \text{Now we will evaluate the right \& left bounds for } U_n\;\&\;U_k:\\
                \text{Right bound: }4-n>4-k\Leftrightarrow n<k\\ 
                \text{Left bound: }9+n < 9+k\Leftrightarrow n < k\\
                \text{Because the right bound }  U_n \text{ is always greater than the } k \text{ variant}\\
                \text{and the respective left bound is less than the } k \text{ variant, for any } n \text{  and } k,\\
                U_n \subset U_k. \\
                \text{And because the intersection of any set and its subset is equal to the subset, }\\
                U_n\cap U_k = U_n\\
                \text{It follows that for the minimum } n \text{ in } U_n,\; U_{n_\text{min}}\subseteq U_n\\ \text{because for any } n \in \N \text{ less than another integer, } U_n \text{ is a subset of said integer's set.} \\
                \text{So }\bigcap_{n=1}^\infty U_n = U_{n_\text{min}}. \text{ Because the minimum of } n \text{ is } 1,\\
                \text{we can write }A=\bigcap_{n=1}^\infty U_n=U_{n_\text{min}}=(3,10).\\
                \text{Therefore } A \text{ is a } G_\delta \text{-set}.\;\qed
            \end{gather*}
		

        \pagebreak
		
		
	\item Show that $[3,10]$ is a $G_\delta$-set.\\
                $Proof:$ Let $A = (3,10)$. WTS $A$ can be written as $A =\bigcap_{n=1}^\infty U_n$
		\begin{gather*}
		    \text{Let } U_n = (3-\frac{1}n, \; 10+\frac{1}n) \\
                \text{Let }k\in \N, \text{ s.t. } k<n. \\
                \text{Now we will evaluate the right \& left bounds for } U_n\;\&\;U_k:\\
                \text{Right bound: }3-\frac{1}n>3-\frac{1}k\Leftrightarrow n<k\\ 
                \text{Left bound: }10+\frac{1}n > 10+\frac{1}k\Leftrightarrow n< k\\
                \text{Because the right bound }  U_n \text{ is always greater than the } k \text{ variant}\\
                \text{and the respective left bound is less than the } k \text{ variant, for any } n \text{  and } k,\\
                U_n \subset U_k. \\
                \text{And because the intersection of any set and its subset is equal to the subset, }\\
                U_n\cap U_k = U_n\\
                \text{It follows that for all } n \text{ in } U_n,\; \lim_{n\rightarrow \infty} U_n \subseteq U_n,\\
                \text{So } U_\infty \cap U_n = U_\infty, \text{which follows that } \bigcap_{n=1}^\infty U_n = \lim_{n\rightarrow \infty} U_n.\\
                \text{So we can write }A=\bigcap_{n=1}^\infty U_n=\lim_{n\rightarrow \infty} U_n=[3,10].\\
                \text{Therefore } A = [3,10] \text{ is a } G_\delta \text{-set}.\;\qed
		\end{gather*}
		
		
		\item Show that $\{1\}$ is a $G_\delta$-set.\\
		$Proof:$ Let $A = {1}$. WTS $A$ can be written as $A =\bigcap_{n=1}^\infty U_n$ 

        
            \begin{gather*}
                \text{Let } U_n = (1-\frac{1}n, \; 1+\frac{1}n) \\
                \text{Let }k\in \N, \text{ s.t. } k<n. \\
                \text{Now we will evaluate the right \& left bounds for } U_n\;\&\;U_k:\\
                \text{Because the right bound }  U_n \text{ is always greater than the } k \text{ variant}\\
                \text{and the respective left bound is less than the } k \text{ variant, for any } n \text{  and } k,\\
                U_n \subset U_k. \\
                \text{And because the intersection of any set and its subset is equal to the subset, }\\
                U_n\cap U_k = U_n\\
                \text{It follows that for all } n \text{ in } U_n,\; \lim_{n\rightarrow \infty} U_n \subseteq U_n,\\
                \text{So } U_\infty \cap U_n = U_\infty, \text{which follows that } \bigcap_{n=1}^\infty U_n = \lim_{n\rightarrow \infty} U_n.\\
                \text{So we can write }A=\bigcap_{n=1}^\infty U_n=\lim_{n\rightarrow \infty} U_n=[1,1] = \{1\}.\\
                \text{Therefore } A = \{1\} \text{ is a } G_\delta \text{-set}.\;\qed
		\end{gather*}
	\pagebreak
		
	\item Is $\emptyset$ a $G_\delta$-set? Why or why not?\\
		$Proof:$ Let $A = \emptyset$. WTS $A$ can be written as $A =\bigcap_{n=1}^\infty U_n$

        \begin{gather*}
            \text{First, we need to establish that }\emptyset\in \R.\\
            \text{Let } U_n=(n,n).\\
            \text{The open interval between any integer and itself is } =\emptyset.\\
            \text{Therefore } U_n=\emptyset.\\
            \text{Additionally, the intersection between the empty sets and all}\\
            \text{other sets is equivalent to the empty set, so therefore for any }n,\\
            U_{n_1}\cap U_{n_2}=\emptyset\cap U_{n_2}=\emptyset.\\
            \text{Hence, } \bigcap_{n=1}^\infty U_n = \emptyset.\\
            \text{Therefore, } A=\bigcap_{n=1}^\infty U_n = \emptyset.\\
            A=\emptyset \text{ is a } G_\delta \text{-set.}\;\qed
        \end{gather*}
		

	\end{enumerate}


\end{document}

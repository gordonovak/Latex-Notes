% --------------------------------------------------------------
% This is all preamble stuff that you don't have to worry about.
% Head down to where it says "Start here"
% --------------------------------------------------------------
 
\documentclass[12pt]{article}
 
\usepackage[margin=1in]{geometry} 
\usepackage{amsmath,amsthm,amssymb,scrextend}
\usepackage{fancyhdr}

\pagestyle{fancy}

 
\newcommand{\N}{\mathbb{N}}
\newcommand{\Z}{\mathbb{Z}}
\newcommand{\I}{\mathbb{I}}
\newcommand{\R}{\mathbb{R}}
\newcommand{\Q}{\mathbb{Q}}
\renewcommand{\qed}{\hfill$\square$}
\let\newproof\proof
\renewenvironment{proof}{\begin{addmargin}[1em]{0em}\begin{newproof}}{\end{newproof}\end{addmargin}\qed}
% \newcommand{\expl}[1]{\text{\hfill[#1]}$}
 
\newenvironment{theorem}[2][Theorem]{\begin{trivlist}
\item[\hskip \labelsep {\bfseries #1}\hskip \labelsep {\bfseries #2.}]}{\end{trivlist}}
\newenvironment{lemma}[2][Lemma]{\begin{trivlist}
\item[\hskip \labelsep {\bfseries #1}\hskip \labelsep {\bfseries #2.}]}{\end{trivlist}}
\newenvironment{problem}[2][Problem]{\begin{trivlist}
\item[\hskip \labelsep {\bfseries #1}\hskip \labelsep {\bfseries #2.}]}{\end{trivlist}}
\newenvironment{exercise}[2][Exercise]{\begin{trivlist}
\item[\hskip \labelsep {\bfseries #1}\hskip \labelsep {\bfseries #2.}]}{\end{trivlist}}
\newenvironment{reflection}[2][Reflection]{\begin{trivlist}
\item[\hskip \labelsep {\bfseries #1}\hskip \labelsep {\bfseries #2.}]}{\end{trivlist}}
\newenvironment{proposition}[2][Proposition]{\begin{trivlist}
\item[\hskip \labelsep {\bfseries #1}\hskip \labelsep {\bfseries #2.}]}{\end{trivlist}}
\newenvironment{corollary}[2][Corollary]{\begin{trivlist}
\item[\hskip \labelsep {\bfseries #1}\hskip \labelsep {\bfseries #2.}]}{\end{trivlist}}
\usepackage{amsthm}
\usepackage{enumitem}
\newtheorem{definition}{Definition}
\begin{document}
 
% --------------------------------------------------------------
%                         Start here
% --------------------------------------------------------------

\lhead{Math 252}
\chead{Ordered Integral Domains}
\rhead{\today}
 
% \maketitle

\newcommand{\n}{\vspace{0.15cm}\\}
\section{Preliminaries}
\begin{definition}$[R^+]$\\
    Let $R$ be an integral domain. A subset $R^+\subseteq R$ denotes the set of \textbf{positive} elements of $R$ such that for all $r\in R^+$, $r > 0_R$. 
\end{definition}

\begin{definition} $[\text{Ordered Integral Domain}]$\\
    Let $R$ be an integral domain. $R$ is said to be ordered if:
    \begin{enumerate}[itemsep=0pt, parsep=0pt]
        \item [P1] If $a$ and $b$ are in $R^+$, then $a+b$ and $ab$ are in $R^+$.
        \item [P2] For all $a\in R$, exactly one of $a\in R^+$, $a=0$, or ${-}a\in R^+$ holds. 
    \end{enumerate}
\end{definition}

\begin{proposition}{1}
    1 is the least element of $R^+$.\vspace{0.1cm}
    \begin{proof}
        Let $c$ be the least element of $R^+$.\n
        Then, we have that:
        \begin{align*}
            c = 1, c> 1, \text{ or } c < 1 \text{ must hold.}
        \end{align*}
        Consider that if $c$ is the least element, $c$ cannot be greater than 1, because $1\in R^+$. \n
        Then, consider that if $0<c<1$, that $0<c^2<c<1$ must also hold. \vspace{-0.1cm}\\
        However, $c^2\in R^+$, and that contradicts that $c$ is the least element in $R^+$.\n
        Thus, 1 is the least element in $R^+$. 
    \end{proof}
\end{proposition}

\begin{proposition}{2}
    $R^+ = \{k1\mid k\in \Z^+\}$.\vspace{0.1cm}
    \begin{proof}
        We will show via induction that $k1\in R^+$ for all $k\in \Z^+$. \n
        \textbf{Base Case: }It is true that if $k=1$, $k1 = 1\in R^+$. \n
        \textbf{Inductive Step: }Suppose that $k1\in R^+$. We have that $(k+1)=k\cdot 1 + 1$ Because $k1$ and $1$ are both in $R^+$, $k1+ 1\in R^+$.  \n
        Therefore, $\{k1\mid k\in \Z^+\}\subseteq R^+$. \n
        Now we will show that $R^+\subseteq \{k1\mid k\in \Z^+\}$. \\
        Let $d$ be the least member of $\{r\in R^+\mid r\ne k1\text{ for all }k\in \Z^+\}$. Because $d\in R^+$, then $d\ge 1$ via \textbf{Proposition 1}.\n
        However, if $d> 1$, then $d-1\in R^+$, and $d$ is the least element, a contradiciton.\\
        Additionally, $d$ cannot equal 1, because $1\in \{k1 \mid k\in \Z^+\}$, a contradiction.\n
        Thus, $\{r\in R^+\mid r\ne k1\text{ for all }k\in \Z^+\}$ does not have a least element, so it has no elements. Therefore, for every $r\in R^+$, there exists an equivalent $k1$, and therefore, $R^+\subseteq \{k1\mid k\in \Z^+\}$.\n
        Hence, $R^+ = \{k1\mid k\in \Z^+\}$.

    \end{proof}
\end{proposition}
\pagebreak
\section*{Theorem}
\begin{theorem}{1}
Let $R\ne 0$ be a well-ordered integral domain. Then an isomorphism $\sigma:\Z\to R$ exists such that if $k<m$ in $\Z$, then $\sigma(k) < \sigma(m) $ are $ R$. 
\end{theorem} 
\begin{proof}
    Define $\sigma:\Z\to R$ by $\sigma(k) = k1$. \\
    We want to show that $\sigma$ is a bijective homomorphism.\n
    We have by algebra that:
    \begin{gather*}
        \sigma(k+m)=\sigma(k)+\sigma(m) \text{ and, }\\
        \sigma(km)=\sigma(k)\cdot\sigma(m) \text { for all }k,m\in \Z.
    \end{gather*}
    Furthermore, we have that $k<m$ implies that $\sigma(k) < \sigma(m)$, because via \textbf{Prop. 2}:
    \begin{align*}
        \sigma(m)-\sigma(k)=(m-k)1\in R^+.
    \end{align*}
    To prove that $\sigma$ is one-to-one, let $\sigma(k) = \sigma(m)$. \\
    Then $(k-m)1 = 0\notin R^+$, so $k\le m$ by \textbf{Prop. 2}. \\
    However, we also have that $(m-k)1=-(k-m)1=0$, so $k\ge m$.\\
    Hence, because $k\le m$ and $k\ge m$, $k=m$, and $\sigma$ is one-to one.\n
    To prove that $\sigma$ is onto, consider an $r\in R$. There are three cases: (i) $r=0,$ (ii) $r>0,$ and (iii) $r<0$.
    \begin{enumerate}[itemsep=0pt, parsep=0pt]
        \item [(i)]If $r=0$, then $r= \sigma(0)$.
        \item [(ii)]If $r>0$, then $r = \sigma(k)$ for some $k\in \Z^+$ (by Prop. 2)
        \item [(iii)]If $r<0$, then $-r>0$, so $r=\sigma(-k)$ for $k\in \Z^+$.
    \end{enumerate} 
    Thus, $\sigma$ is onto as required. \n
    Hence, $\sigma$ is an isomorphism from $\Z \to R$, such that if $k<m$, then $\sigma(k)<\sigma(m)$. 
\end{proof}
 
 
% --------------------------------------------------------------
%     You don't have to mess with anything below this line.
% --------------------------------------------------------------
 
\end{document}
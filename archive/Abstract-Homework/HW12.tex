%document class
\documentclass[10pt,twoside]{article}

%{
%packages
\usepackage[top=1in,bottom=0.6in,left=1in,right=1in]{geometry}
\usepackage{latexsym}
\usepackage{amssymb}
\usepackage{amsfonts}
\usepackage{amstext}
\usepackage{amsmath}
\usepackage{amsthm}
\usepackage{multicol}
\usepackage{hyperref}
\usepackage{enumerate}
\usepackage{tikz}
\usepackage{pgfplots}
\usepackage{array}
\usepackage{fancyhdr}
\usepackage{xcolor, mdframed}
\usepackage{enumitem}
\pgfplotsset{
    humanaxes/.style={axis lines=center, every axis plot/.append style={very thick, mark size=3}, x axis line style=-, y axis line style=-},
    humanaxeslabels/.style={every axis x label/.style={at={(current axis.right of origin)},anchor=west},every axis y label/.style={at={(current axis.above origin)},anchor=south}},
    human/.style={humanaxes, humanaxeslabels}
}
\pgfplotsset{compat=1.16} %added beacuse of some updated tex error


%if you want to remove page numbers
%\pagestyle{empty}


%theorems
\theoremstyle{plain}
\newtheorem{Theorem}{Theorem}
\newtheorem{Proposition}[Theorem]{Proposition}
\newtheorem{Corollary}[Theorem]{Corollary}
\newtheorem{Lemma}[Theorem]{Lemma}
\newtheorem{Question}[Theorem]{Question}
\newtheorem{Conjecture}[Theorem]{Conjecture}
\newtheorem{Assumption}[Theorem]{Assumption}
\newtheorem{Algorithm}[Theorem]{Algorithm}

\theoremstyle{definition}
\newtheorem{Definition}[Theorem]{Definition}
\newtheorem{Property}[Theorem]{Property}
\newtheorem{Notation}[Theorem]{Notation}
\newtheorem{Condition}[Theorem]{Condition}
\newtheorem{Example}[Theorem]{Example}
\newtheorem{Exercise}[Theorem]{Exercise}
\newtheorem{Introduction}[Theorem]{Introduction}
\theoremstyle{remark}
\newtheorem{Remark}[Theorem]{Remark}



%bold topics
\newcommand\topic[1]{\noindent{\bf #1}}

%definition in a box with color
\newcommand{\defn}[1]{
\begin{mdframed}[backgroundcolor=blue!05] #1
\end{mdframed}
}

%hint command
\newcommand{\hint}[1]{\noindent{\footnotesize {\it #1}}}

% here is highlighted/colored text
\newcommand{\hl}[1]{\textcolor{red}{#1}} %note that \hl{} highlights text like a highlighter
\newcommand{\hlred}[1]{\textcolor{red}{#1}}
\newcommand{\hlblue}[1]{\textcolor{blue}{#1}}
\newcommand{\hlgreen}[1]{\textcolor{green}{#1}}
\newcommand{\mathhl}[1]{\colorbox{yellow}{$#1$}}


%This will put a circle around something.
\newcommand*\circled[1]{\tikz[baseline=(char.base)]{
            \node[shape=circle,draw,inner sep=2pt] (char) {#1};}}

%These are two other examples of matrices.
%$G = \bigg\{ \begin{pmatrix} a & b \\ 0 & a \end{pmatrix} \bigg| a,b \in \mathbb{R}, a\neq 0 \bigg\}$ 
%$G = \bigg\{ \begin{bmatrix} a & b \\ 0 & a \end{bmatrix} \bigg| a,b \in \mathbb{R}, a\neq 0 \bigg\}$ 


% Commands for abstract algebra

\newcommand{\integers}{\mathbb{Z}}
\newcommand{\reals}{\mathbb{R}}
\newcommand{\complex}{\mathbb{C}}
\newcommand{\normal}{\triangleleft}
\newcommand{\rationals}{\mathbb{Q}}
\newcommand{\field}{\mathbb{F}}
\newcommand{\naturals}{\mathbb{N}}

\newcommand{\aut}[1]{{\rm Aut}(#1)}
\newcommand{\Ker}{{\rm Ker}\,}
\newcommand{\Ima}{{\rm Im}\,}
\newcommand{\cyclic}[1]{\langle #1 \rangle}
\newcommand{\isom}{\cong}

\newcommand{\NN}{\mathbb{N}}
\newcommand{\ZZ}{\mathbb{Z}}
\newcommand{\QQ}{\mathbb{Q}}
\newcommand{\RR}{\mathbb{R}}
\newcommand{\CC}{\mathbb{C}}
\newcommand{\FF}{\mathbb{F}}

\newcommand{\N}{\mathbb{N}}
\newcommand{\Z}{\mathbb{Z}}
\newcommand{\Q}{\mathbb{Q}}
\newcommand{\R}{\mathbb{R}}
\newcommand{\C}{\mathbb{C}}
\newcommand{\F}{\mathbb{F}}

% my commands
\newcommand{\vp}{\vspace{0.15cm}\\}
\newcommand{\vpp}{\vspace{0.25cm}\\}
\newcommand{\vpn}{\vspace{0.05cm}\\}
\newcommand{\twom}[4]{\begin{bmatrix}#1 & #2 \\ #3 & #4\end{bmatrix}}
\newcommand{\rmv}[1]{\,\backslash\{#1\}}
\newcommand{\casi}[4]{\begin{cases}#1 & \text{ if }#2\\ #3 & \text{ if }#4\end{cases}}
\newcommand{\ang}[1]{\langle #1 \rangle}
\usepackage{changepage}
\usepackage{array}

%if you want to change some counters you can use the command below
%\setcounter{section}{1}

\begin{document}
\section*{Homework 12}

\pagestyle{fancy}
\lhead{MATH 252}
\chead{\large{\textbf{Homework 12}}}
\rhead{Book section: 4.3}
\lfoot{}
\cfoot{}
%\rfoot{\thepage/\pageref{LastPage} }
\setlength{\headheight}{14pt} %added in bc warning

%\setcounter{section}{1}
\subsection*{From the textbook, Section 4.3}
\begin{enumerate}
    %%%%%%%%%% Exercise 2
    \item \underline{Do Exercise 2 Parts (a) and (b). Neither of these rings is a field... Why?}\vp
    In each case describe $R=F[x]/\ang{h}$ as in Theorem 2 and write out the addition and multiplication tables for $R$. 
    \begin{enumerate}
        %%%%%%%%%%% Part a
        \item \underline{$h= x^2+1, F= \Z_2$.}
        \begin{center}\newcommand{\eq}{$x^2+1$}\newcommand{\xo}{$x+1$}\newcommand{\ex}{$x$}\begin{tabular}{c c}
        \noindent\begin{tabular}{c | c c c c }
            + & 0   & 1     & $x$     & \xo  \\
            \cline{1-5}
            0 & 0       & 1     & \ex   & \xo \\
            1 & 1       & 0     & \xo   & \ex \\
            \ex & \ex   & \xo   & 0   & 1 \\
            $x+1$ & \xo & \ex   & 1     & 0
        \end{tabular}
        &
        \begin{tabular}{c | c c c c }
            $\cdot$ & 0   & 1     & \ex     & \xo  \\
            \cline{1-5}
            0 & 0       & 0     & 0   & 0 \\
            1 & 0       & 1     & \ex   & \xo \\
            \ex & 0   & \ex   & 1   & \xo \\
            $x+1$ & 0 & \xo   & \xo     & 0
        \end{tabular}
        \end{tabular}\end{center}
        This one isn't a field because there is a zero divisor $\rightarrow (x+1)^2$. 
        %%%%%%%%%%%% Part b
        \item \underline{$h= x^2+x, F= \Z_2$}.
        \begin{center}\newcommand{\eq}{$x^2+1$}\newcommand{\xo}{$x+1$}\newcommand{\ex}{$x$}\begin{tabular}{c c}
        \noindent\begin{tabular}{c | c c c c }
            + & 0   & 1     & $x$     & \xo  \\
            \cline{1-5}
            0 & 0       & 1     & \ex   & \xo \\
            1 & 1       & 0     & \xo   & \ex \\
            \ex & \ex   & \xo   & 0   & 1 \\
            $x+1$ & \xo & \ex   & 1     & \ex
        \end{tabular}
        &
        \begin{tabular}{c | c c c c }
            $\cdot$ & 0   & 1     & \ex     & \xo  \\
            \cline{1-5}
            0 & 0       & 0     & 0   & 0 \\
            1 & 0       & 1     & \ex   & \xo \\
            \ex & 0   & \ex   & \ex   & 0 \\
            $x+1$ & 0 & \xo   & 0     & \xo
        \end{tabular}
        \end{tabular}\end{center}
        This one also isn't a field because there's a zero divisors $\rightarrow x(x+1)=0$. 
    \end{enumerate}

    %%%%%%%%%%%%%%%%%%%%
    %%%%%%%%% Exercise 3
    \item \underline{Do Exercise 3. Be inspired by Example 7!}\vpp
    Construct a field of order 8 and write down the multiplication table.
    \begin{adjustwidth}{0.5cm}{0.5cm}
        Consider the polynomial $f(x) = x^3+x+1$. It has no root in $Z_2$, as $f(0)= 1$ and $f(1)=1$, so it is irreducible. \\
        Consider the field:
        \begin{align*}
            F=\frac{\Z_2[x]}{\ang{x^3+x+1}}=\{a+bt+ct^2\mid a,b,c\in \Z_2\}
        \end{align*}
        We know it's a field because $f(x)$ is irreducible (and looking at the mult. table below reveals that as well).
    \end{adjustwidth}
    \begin{adjustwidth}{-1cm}{0cm}
        \begin{center}\newcommand{\eq}{$x^2+1$}\newcommand{\xo}{$x+1$}\newcommand{\ex}{$x$}\newcommand{\xt}{$x^2$}\newcommand{\xtx}{$x^2+x$}\newcommand{\xto}{$x^2+1$}\newcommand{\xtxo}{$x^2+x+1$}
            \begin{tabular}{c c}
        \begin{tabular}{c | c c c c c c c c}
            $\cdot$ & 0   & 1     & \ex     & \xo & $x^2$ & $x^2+1$ & $x^2+x$ & $x^2+x+1$ \\
            \cline{1-9}
            0 & 0       & 0     & 0   & 0 & 0 & 0& 0& 0\\
            1 & 0       & 1     & \ex       & \xo  & \xt   & \xto   & \xtx  & \xtxo \\
            \ex & 0   & \ex     & \xt       & \ex  & \xt   & 1      & \xtxo & \xt \\
            \xo & 0 & \xo       & \xtx      & \xto & \xtxo & \xt    & 1     & \ex   \\
            \xt     &0 &\xt     & \xo       & \xtxo& \xtx  & \ex    & \xto  & 1       \\
            \xto    &0&\xto     &  1        & \xt  & \ex   & \xtxo  & \xo   & \xtx \\
            \xtx    &0&\xtx     &  \xtxo    & 1    & \xto  & \xo    & \ex   & \xt        \\
            \xtxo   &0&\xtxo    &  \xt      & \ex  & 1     & \xtx   & \xt   & \xo
        \end{tabular}
        \end{tabular}\end{center}
    \end{adjustwidth}
        
    
    \item \underline{Do Exercise 7. Hint: the inverse is a quadratic... so multiply by an arbitrary quadratic, set equal to 1}\vp\underline{and find equations for the coefficients of the inverse.}\vpp
    In each case show that $r$ is a unit in $R=F[x]\ang{h}$ and exhibit the inverse. \\Use the notation of Theorem 2. 
    \begin{enumerate}
        %%%%%% part 1
        \item $r=1+t^2,\;F=\Z_{11},\;h=x^3+1$.\vp
        Consider that $(1+t^2)(at^2+bt+c)=1$. 
        \begin{align*}
            (1+x^2)(ax^2+bx+c)&=ax^2+bx+c+ax^4+bx^3+cx^2\\
            &=ax^4+bx^3+(a+c)x^2+bx+c.
        \end{align*}
        Now consider $axh + dh$ as a divisor of this field. We get as a result that:\\
        $c+10b = 1$, $b+10a = 0$, and $c+a=0$. Solving the system we find that:
        \begin{align*}
            a=5\\
            b=5\\
            c=6
        \end{align*}
        So the inverse of $r$ is $5x^2+5x+6$. 
        %%%%%% part 2
        \item $r=1+t-t^2, \;F=\Z_7, h=x^3+x^2-1$\vp
        First, let simplify $r$ to $1+x+6x^2$, and $h$ to $x^3+x^2+6$.\vp
        Now we use a quadratic to model this scenario once again because the order of our $h$ is 3, so our possible unit must be less than that:
        \begin{align*}
            (1 +x-x^2)(a+bx+cx^2)&=a(1+x-x^2)+bx(1+x-x^2)+cx^2(1+x-x^2)\\
            &=a+ax-ax^2+bx+bx^2-bx^3+cx^2+cx^3-cx^4\\
            &=a+(a+b)x+(-a+b+c)x^2+(-b+c)x^3-cx^4
        \end{align*}
        So we consider a $h$ and try to get it to reduce this polynomial. For example, $-c\cdot x\cdot h$ gets rid of the first term but leaves us with $(-b+2c)x^3$ and $(a+b-c)x$. So we multiply out $h$ again by $(-b+2c)h$ and continue on as follows. Eventually we get an $a,b,c$ such that:
        \begin{align*}
            a-b-5c&=1\\
            a+b-c&=0\\
            -a-5b+6c&=0
        \end{align*}
        Solving the system of equations yeilds:
        \begin{align*}
            a=3\\
            b=6\\
            c=2
        \end{align*}
        Which gives us an inverse of $3+6x+2x^2$.
    \end{enumerate}

    \item (Not from the book) Construct a field of order 27. Explain why your construction is a field and why it has the correct number of elements.\vp
    I'm just going to create an irreducible polynomial of degree 9 in $\Z_3$, and then use my polynomial $f(x)$ such that:
    \begin{align*}
        \frac{\Z_3[x]}{f(x)} \text{ is a field.}
    \end{align*}
    We know this will be a field of order 27 if and only if our polynomial is of degree 9, because then we have $3\cdot 9 - 1=26$ elements that are not divisible by $f(x)$, giving us 27 total elements \vp
    I chose $\Z_3$ because it's easy to test for roots and other divisors.
    We'll try:
    \begin{align*}
        f(x)=2x^9 + x + 1
    \end{align*}
    Consider that $f(1) = 1$ and $f(0)= 1$ and $f(2)= 1$. Thus, $f(x)$ has no linear factors. It also doesn't have other factors apart from that (I believe). 
    \begin{align*}
        \frac{\Z_3[x]}{2x^9 + x + 1} \text{ is a field!}
    \end{align*}
\end{enumerate}
\newpage

\begin{center}
    \begin{tabular}{c | c}
        $x^3+x^2+0+1$ & $x^2+1$\\
        \cline{1-2}
        $x^3+x^2+0+1$\\
        $-(x^3+x+0+0)$ &  $x$\\
        \cline{1-2}
        $x^2+x+1$ \\ 
        $-(x^2+1)$ & $1$\\
        \cline{1-2}
        $x$\\
        $\text{R}=x$ & $\text q = x+1$
    \end{tabular}
\end{center}


\end{document}
%document class
\documentclass[10pt,twoside]{article}

%packages
\usepackage[top=1in,bottom=0.6in,left=1in,right=1in]{geometry}
\usepackage{latexsym}
\usepackage{amssymb}
\usepackage{amsfonts}
\usepackage{amstext}
\usepackage{amsmath}
\usepackage{amsthm}
\usepackage{multicol}
\usepackage{hyperref}
\usepackage{enumerate}
\usepackage{tikz}
\usepackage{pgfplots}
\usepackage{fancyhdr}
\usepackage{xcolor, mdframed}
\pgfplotsset{
	humanaxes/.style={axis lines=center, every axis plot/.append style={very thick, mark size=3}, x axis line style=-, y axis line style=-},
	humanaxeslabels/.style={every axis x label/.style={at={(current axis.right of origin)},anchor=west},every axis y label/.style={at={(current axis.above origin)},anchor=south}},
	human/.style={humanaxes, humanaxeslabels}
}
\pgfplotsset{compat=1.16} %added beacuse of some updated tex error


%if you want to remove page numbers
%\pagestyle{empty}

%bold topics
\newcommand\topic[1]{\noindent{\bf #1}}

%definition in a box with color
\newcommand{\defn}[1]{
\begin{mdframed}[backgroundcolor=blue!05] #1
\end{mdframed}
}

%hint command
\newcommand{\hint}[1]{\noindent{\footnotesize {\it #1}}}

%theorems
\theoremstyle{plain}
\newtheorem{Theorem}{Theorem}
\newtheorem{Proposition}[Theorem]{Proposition}
\newtheorem{Corollary}[Theorem]{Corollary}
\newtheorem{Lemma}[Theorem]{Lemma}
\newtheorem{Question}[Theorem]{Question}
\newtheorem{Conjecture}[Theorem]{Conjecture}
\newtheorem{Assumption}[Theorem]{Assumption}
\newtheorem{Algorithm}[Theorem]{Algorithm}

\theoremstyle{definition}
\newtheorem{Definition}[Theorem]{Definition}
\newtheorem{Property}[Theorem]{Property}
\newtheorem{Notation}[Theorem]{Notation}
\newtheorem{Condition}[Theorem]{Condition}
\newtheorem{Example}[Theorem]{Example}
\newtheorem{Exercise}[Theorem]{Exercise}
\newtheorem{Introduction}[Theorem]{Introduction}
\theoremstyle{remark}
\newtheorem{Remark}[Theorem]{Remark}


\newcommand{\mymk}[1]{%
  \tikz[baseline=(char.base)]\node[anchor=south west, draw,rectangle, rounded corners, inner sep=2pt, minimum size=7mm,
    text height=2mm](char){\ensuremath{#1}} ;}



% here is highlighted/colored text
\newcommand{\hl}[1]{\textcolor{red}{#1}} %note that \hl{} highlights text like a highlighter
\newcommand{\hlred}[1]{\textcolor{red}{#1}}
\newcommand{\hlblue}[1]{\textcolor{blue}{#1}}
\newcommand{\hlgreen}[1]{\textcolor{green}{#1}}
\newcommand{\mathhl}[1]{\colorbox{yellow}{$#1$}}


%This will put a circle around something.
\newcommand*\circled[1]{\tikz[baseline=(char.base)]{
            \node[shape=circle,draw,inner sep=2pt] (char) {#1};}}

%These are two other examples of matrices.
%$G = \bigg\{ \begin{pmatrix} a & b \\ 0 & a \end{pmatrix} \bigg| a,b \in \mathbb{R}, a\neq 0 \bigg\}$ 
%$G = \bigg\{ \begin{bmatrix} a & b \\ 0 & a \end{bmatrix} \bigg| a,b \in \mathbb{R}, a\neq 0 \bigg\}$ 


% Commands for abstract algebra

\newcommand{\integers}{\mathbb{Z}}
\newcommand{\reals}{\mathbb{R}}
\newcommand{\complex}{\mathbb{C}}
\newcommand{\normal}{\triangleleft}
\newcommand{\rationals}{\mathbb{Q}}
\newcommand{\field}{\mathbb{F}}
\newcommand{\naturals}{\mathbb{N}}

\newcommand{\aut}[1]{{\rm Aut}(#1)}
\newcommand{\Ker}{{\rm Ker}\,}
\newcommand{\Ima}{{\rm Im}\,}
\newcommand{\cyclic}[1]{\langle #1 \rangle}
\newcommand{\isom}{\cong}

\newcommand{\NN}{\mathbb{N}}
\newcommand{\ZZ}{\mathbb{Z}}
\newcommand{\QQ}{\mathbb{Q}}
\newcommand{\RR}{\mathbb{R}}
\newcommand{\CC}{\mathbb{C}}
\newcommand{\FF}{\mathbb{F}}

%if you want to change some counters you can use the command below
%\setcounter{section}{1}

\pagestyle{fancy}
\lhead{MATH 252}
\chead{\large{\textbf{Daily Prep 01} }}
\rhead{Book section: 0.1, 0.2}
\lfoot{}
\cfoot{}
%\rfoot{\thepage/\pageref{LastPage} }
\setlength{\headheight}{14pt} %added in bc warning


\begin{document}


\section*{Daily Prep 03}
%%commands for fancy headers
\pagestyle{fancy}
\lhead{MATH 252}
\chead{\large{\textbf{Daily Prep 03} }}
\rhead{Book section: 0.4, 1.1}
\lfoot{}
\cfoot{}
%\rfoot{\thepage/\pageref{LastPage} }
\setlength{\headheight}{14pt} %added in bc warning




\subsection*{Read section 0.4}


\begin{enumerate}


\item Consider the definition of \textbf{equivalence} (or equivalence relation) (p.~17). 
Match the three descriptions of the properties a \textit{relation} needs to have to be an equivalence  with their proper names. 

\begin{tabular}{l   l}
(a)  If $x$ is equivalent  to $y$, then $y$ is equivalent to $x$.  \hspace{1in} & (i) reflexive property  \\
(b) An element must be equivalent to itself. & (ii) symmetric property \\
(c) If $x$  is equivalent  to $y$ and $y$  is equivalent  to $z$,  & (iii) transitive property \\
\ \ \ \ \   then $x$ must be equivalent to $z$. & \\
\end{tabular}\\
\begin{enumerate}
    \item $\rightarrow$ symmetric property
    \item $\rightarrow$ reflexive property
    \item $\rightarrow$ transitive property
\end{enumerate}



\item Each of the following relations fails to be an equivalence relation.  In each case, which of the three properties (\textbf{reflexive, symmetric, transitive}) does the relation fail to satisfy?
	\begin{enumerate}
	\item For $a,b\in \reals$, say $a$ and $b$ are related if  $a<b$.
        \begin{enumerate}
            \item It fails for \textbf{the reflexive property}
        \end{enumerate}
\item For $a,b\in \reals$, say $a$ and $b$ are related if  $a\le b$.
\begin{enumerate}
    \item It fails for \textbf{the symmetric property}
\end{enumerate}
\item For $x$ and $y$ in the set of all humans, say $x$ and $y$ are related if they share at least one biological parent. 
    \begin{enumerate}
        \item It fails for \textbf{the transitive property}
    \end{enumerate}
	
	\end{enumerate}
	
\item Consider the definition of \textbf{equivalence class} (p.~17).  If we say that the humans $x$ and $y$ are related iff they share \textit{both} biological parents, then we get an equivalence relation.   List the people in your own equivalence class.
\begin{enumerate}
    \item Me 
    \item My brother
\end{enumerate}
\item Consider Example 6. The ``shapes" of the \textbf{partitions} are 1 (for $\{A\}$), 2-1 (for each of $\{\{1,2\},\{3\}\}$, $\{\{1,3\},\{2\}\}$, and $\{\{2,3\},\{1\}\}$), and 1-1-1 (for $\{\{1\},\{2\},\{3\}\}$). (Note that the shape 2-1 is the same as the shape 1-2.) What are the different shapes for partitions of $B=\{a,b,c,d\}$? List them. 

\begin{enumerate}
    \item \{\{a,b,c,d\}\}
    \item \{\{a,b\},\{c,d\}\}
    \item \{\{a,c\},\{b,d\}\}
    \item \{\{a,d\},\{c,d\}\}
    \item \{\{a\},\{b\},\{c\},\{d\}\}
\end{enumerate}

\item If $a,b\in\ZZ$ then say that $a$ and $b$ are related iff they have the same remainder after division by 3.  It is a fact that this is an equivalence relation.  	
	\begin{enumerate}
	\item Are $3$ and 10 related?
        \begin{enumerate}
            \item No
        \end{enumerate}
	\item List three integers that are related to 3, and three that are related to 10.
	\begin{enumerate}
        \item $3\equiv 0,6,9$
        \item $10\equiv 1,4,7$
    \end{enumerate}
	\item Describe the equivalence class of 0.
	\begin{enumerate}
        \item All integers divisible by 3
    \end{enumerate}
	\end{enumerate}
	
\end{enumerate}



\subsection*{Start reading section 1.1}

\begin{enumerate}
 \setcounter{enumi}{5}
\item Consider the {\bf{Principle of Mathematical Induction}} on page 24.  In order to prove a statement is true for all $n \geq 1$, which part of the principle, (1) or (2), requires the {\bf{induction hypothesis}} (also known as the \textit{inductive hypothesis})?
\begin{enumerate}
    \item Part 2, where we assume that $p_k$ is true.
\end{enumerate}
\item Read Example 1. State $p_5$ and verify it.
\begin{enumerate}
    \item $p_5:$ Must equal $\frac{1}2k(k+1)=15$
    \item $1+2+3+4+5=15$.
\end{enumerate}
    
\end{enumerate}


\medskip

    \hint{ \textbf{Stuck?} Ask a question in the \textbf{Forum} on Moodle.}

\newpage
\end{document}
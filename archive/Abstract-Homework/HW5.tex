 %document class
\documentclass[10pt,twoside]{article}

%{
%packages
\usepackage[top=1in,bottom=0.6in,left=1in,right=1in]{geometry}
\usepackage{latexsym}
\usepackage{amssymb}
\usepackage{amsfonts}
\usepackage{amstext}
\usepackage{amsmath}
\usepackage{amsthm}
\usepackage{multicol}
\usepackage{hyperref}
\usepackage{enumerate}
\usepackage{tikz}
\usepackage{pgfplots}
\usepackage{fancyhdr}
\usepackage{xcolor, mdframed}
\usepackage{enumitem}
\pgfplotsset{
	humanaxes/.style={axis lines=center, every axis plot/.append style={very thick, mark size=3}, x axis line style=-, y axis line style=-},
	humanaxeslabels/.style={every axis x label/.style={at={(current axis.right of origin)},anchor=west},every axis y label/.style={at={(current axis.above origin)},anchor=south}},
	human/.style={humanaxes, humanaxeslabels}
}
\pgfplotsset{compat=1.16} %added beacuse of some updated tex error


%if you want to remove page numbers
%\pagestyle{empty}


%theorems
\theoremstyle{plain}
\newtheorem{Theorem}{Theorem}
\newtheorem{Proposition}[Theorem]{Proposition}
\newtheorem{Corollary}[Theorem]{Corollary}
\newtheorem{Lemma}[Theorem]{Lemma}
\newtheorem{Question}[Theorem]{Question}
\newtheorem{Conjecture}[Theorem]{Conjecture}
\newtheorem{Assumption}[Theorem]{Assumption}
\newtheorem{Algorithm}[Theorem]{Algorithm}

\theoremstyle{definition}
\newtheorem{Definition}[Theorem]{Definition}
\newtheorem{Property}[Theorem]{Property}
\newtheorem{Notation}[Theorem]{Notation}
\newtheorem{Condition}[Theorem]{Condition}
\newtheorem{Example}[Theorem]{Example}
\newtheorem{Exercise}[Theorem]{Exercise}
\newtheorem{Introduction}[Theorem]{Introduction}
\theoremstyle{remark}
\newtheorem{Remark}[Theorem]{Remark}



%bold topics
\newcommand\topic[1]{\noindent{\bf #1}}

%definition in a box with color
\newcommand{\defn}[1]{
\begin{mdframed}[backgroundcolor=blue!05] #1
\end{mdframed}
}

%hint command
\newcommand{\hint}[1]{\noindent{\footnotesize {\it #1}}}

% here is highlighted/colored text
\newcommand{\hl}[1]{\textcolor{red}{#1}} %note that \hl{} highlights text like a highlighter
\newcommand{\hlred}[1]{\textcolor{red}{#1}}
\newcommand{\hlblue}[1]{\textcolor{blue}{#1}}
\newcommand{\hlgreen}[1]{\textcolor{green}{#1}}
\newcommand{\mathhl}[1]{\colorbox{yellow}{$#1$}}


%This will put a circle around something.
\newcommand*\circled[1]{\tikz[baseline=(char.base)]{
            \node[shape=circle,draw,inner sep=2pt] (char) {#1};}}

%These are two other examples of matrices.
%$G = \bigg\{ \begin{pmatrix} a & b \\ 0 & a \end{pmatrix} \bigg| a,b \in \mathbb{R}, a\neq 0 \bigg\}$ 
%$G = \bigg\{ \begin{bmatrix} a & b \\ 0 & a \end{bmatrix} \bigg| a,b \in \mathbb{R}, a\neq 0 \bigg\}$ 


% Commands for abstract algebra

\newcommand{\integers}{\mathbb{Z}}
\newcommand{\reals}{\mathbb{R}}
\newcommand{\complex}{\mathbb{C}}
\newcommand{\normal}{\triangleleft}
\newcommand{\rationals}{\mathbb{Q}}
\newcommand{\field}{\mathbb{F}}
\newcommand{\naturals}{\mathbb{N}}

\newcommand{\aut}[1]{{\rm Aut}(#1)}
\newcommand{\Ker}{{\rm Ker}\,}
\newcommand{\Ima}{{\rm Im}\,}
\newcommand{\cyclic}[1]{\langle #1 \rangle}
\newcommand{\isom}{\cong}

\newcommand{\NN}{\mathbb{N}}
\newcommand{\ZZ}{\mathbb{Z}}
\newcommand{\QQ}{\mathbb{Q}}
\newcommand{\RR}{\mathbb{R}}
\newcommand{\CC}{\mathbb{C}}
\newcommand{\FF}{\mathbb{F}}

\newcommand{\N}{\mathbb{N}}
\newcommand{\Z}{\mathbb{Z}}
\newcommand{\Q}{\mathbb{Q}}
\newcommand{\R}{\mathbb{R}}
\newcommand{\C}{\mathbb{C}}
\newcommand{\F}{\mathbb{F}}

\newcommand{\vp}{\vspace{0.15cm}\\}
\newcommand{\vpp}{\vspace{0.25cm}\\}
\newcommand{\vpn}{\vspace{0.05cm}\\}
\newcommand{\br}[1]{\{ #1 \}}
\newcommand{\brs}[2]{\{ #1 \;|\; #2 \}}

%if you want to change some counters you can use the command below
%\setcounter{section}{1}

\begin{document}

\pagestyle{fancy}
\lhead{MATH 252}

\chead{\large{\textbf{Homework 03} }}
\rhead{Book section: 0.4, 1.1}
\lfoot{}
\cfoot{}
%\rfoot{\thepage/\pageref{LastPage} }
\setlength{\headheight}{14pt} %added in bc warning

%}

\newpage

\pagestyle{fancy}
\lhead{MATH 252}
\chead{\large{\textbf{Homework 05} }}
\rhead{Book section: 2.1, 2.2}
\lfoot{}
\cfoot{}
%\rfoot{\thepage/\pageref{LastPage} }
\setlength{\headheight}{14pt} %added in bc warning

\section*{Homework 05}




\subsection*{From the textbook, Section 2.1}

\begin{enumerate}
    \item From Exercise 1 (a), (c), (d). For each part:
    \begin{itemize}
        \item If the operation is commutative, prove it.  If it is not, then give a concrete counterexample.
        \item If the operation has a unity identify it.  If it does not, then give a proof that it does not (say, by contradiction?).
        \item If the operation has units, then describe them nicely as a set using good notation!
    \end{itemize}  
    \begin{enumerate}
        \item $M=\mathbb{Z}; a*b = a - b$.
        \begin{enumerate}
            \item This is not \textbf{commutative}.
            \begin{align*}
                a - b &= b - a \\
                a &= -a
            \end{align*}
            For all \( \mathbb{Z} \setminus \{0\} \), this does not hold true.
            
            \item It also does not have a \textbf{unity}.
            \begin{align*}
                (a - e) &= (e - a) \\
                a &= e
            \end{align*}
            The unity cannot change from element to element.
        \end{enumerate}
        
        \item SKIP
        \item $M=\R; a*b=a+b-ab$.
        \begin{enumerate}
            \item This operation is \textbf{commutative}.
            \begin{align*}
                a+b-ab &= b+a-ba\\
                -ab &= -ba\\
                1&=1
            \end{align*}
            \item This operation has a \textbf{unity}\\
                \textit{Proof:} Let $e = 0$.
                \begin{align*}
                    a + e - ae &= e + a - ea\\
                    a + 0 - 0 &= a + 0 - 0\qed
                \end{align*}
            \item We want to find $a^{-1}\in M$ such that $a\cdot a^{-1}=e$.\\
                \begin{align*}
                    a + x_a - ax_a &= e\\
                    x_a(1-a) + a &= e\\
                    x_a = \frac{-a}{(1-a)}&
                \end{align*}
                So, $x_a$ above is the inverse of any given element $a$, and $x_a\in M$ because $M=\R$.
        \end{enumerate}
        \item $M=$ any set with $|M|\ge 2;$ $a*b=b$.
        \begin{enumerate}
            \item This is not \textbf{commutative}.
            \begin{align*}
                a*b= b \ne b*a = a
            \end{align*}
            For all $a\ne b$, this does not hold true.
            \item It also does not have a \textbf{unity}.
            \begin{align*}
                (a * e) &= e \ne (e*a) = a
            \end{align*}
            The unity cannot change from element to element.
        \end{enumerate}

    \end{enumerate}
    \pagebreak
    \item Do Exercise 3 (a). Note that there are only four possible ways to fill in the table.  So you can simply consider each one as a separate case.  Also:  if you want some help formatting a table in \LaTeX, check out \url{https://www.overleaf.com/learn/latex/Tables} and \url{https://www.tablesgenerator.com/}
    \begin{enumerate}
        \item Completing table one: \vp
                \begin{tabular}{lll}
                    \multicolumn{1}{l|}{}  & a & b     \\\hline
                    \multicolumn{1}{l|}{a} &   & b     \\
                    \multicolumn{1}{l|}{b} &   & a     \\
                \end{tabular} $\Longrightarrow$
                \begin{tabular}{lll}
                    \multicolumn{1}{l|}{}  & a & b     \\\hline
                    \multicolumn{1}{l|}{a} & a & b     \\
                    \multicolumn{1}{l|}{b} & b & a     \\
                \end{tabular}\vp
                This forces table one to be associative, because $(b*a)*a = (a*a)*b$, and communative, because $b*a = a*b$.
                \item Completing table two:\vp
                \begin{tabular}{lll}
                    \multicolumn{1}{l|}{}  & a & b     \\\hline
                    \multicolumn{1}{l|}{a} &   & a     \\
                    \multicolumn{1}{l|}{b} &   & b     \\
                \end{tabular} $\Longrightarrow$
                \begin{tabular}{lll}
                    \multicolumn{1}{l|}{}  & a & b     \\\hline
                    \multicolumn{1}{l|}{a} & b & a     \\
                    \multicolumn{1}{l|}{b} & a & b     \\
                \end{tabular}\vp
                Now, table two also follow the associative property, because $(a*a)*b=b=a*(a*b)$. This also satisfied communativity as $a*b = b*a$. 
    \end{enumerate}
    \item Do Exercise 9. Careful: do NOT assume that inverses exist since you are ONLY working in a monoid, not a group. But you do have \textit{canceling laws}! \vp
    Assume that $a$ is left cancelable in a moniod $M$.
    \begin{enumerate}
        \item If $a^5=b^5$ and $a^{12} = b^{12}$, show that $a = b$.
        \begin{proof}
            We want to show that $a = b$. \vp
            If $a^{12} = b^{12}$, then $a^5a^5a^2 = b^5b^5b^2$.\\
            Because $a^5= b^5$, then $a^5a^5a^7=a^5a^5b^2$. \\
            Thus, because $a$ is cancelable, $a^2 = b^2$.\vp
            If $a^5 = b^5$, then $a^2a^2 a = b^2b^2 b$. \\
            Because $a^2 = b^2$, $a^2 a^2 a = a^2 a^2 b$. \vpp
            Hence, because $a$ is cancelable, $a = b$. 
        \end{proof}
        \item If $a^m = b^m$ and $a^n=b^n$ where $m$ and $n$ are relatively prime, show that $a=b$.
        \begin{proof}
            We want to show that $a = b$.\vp
            Because $m$ and $n$ are relatively prime, Let $x, y\in \Z$ such that $xm + yn = 1$. \vp
            Becuase $a^m = b^n$ and $a^m=b^m$, then, we write:
            \begin{align*}
                a^{xm}a^{yn}=a^{xm+yn}&=a^1\\
                (a^m)^x(a^n)^y&=a\\
                \text{Substitute in } b^n \;\&\; b^m &\text{ for } a^n \;\&\; a^m: \\
                (b^m)^x(b^n)^y= b^{mx}b^{my}&=b^1=a
            \end{align*}
            Hence, $a=b$. 
        \end{proof}
    \end{enumerate}
    \hint{Hint: $a^{24}a=a^{24}b$ implies $a=b$.}
     
\end{enumerate}


\pagebreak
\subsection*{From the textbook, Section 2.2}

\begin{enumerate}

\setcounter{enumi}{3}

    \item From Exercise 1 (c), (e), (f), (g). If you think the given set-operation pair forms a group, then prove/verify each group axiom.  Otherwise, state explicitly which axioms fail with proof/counterexample.  

    Only argue associativity for part (c). For (e),(f),(g) you do not need to argue associativity since we are using operations which we *know* are already associative!  

    \begin{enumerate}
    \item \textit{Skip}
    \item \textit{Skip}
    \item $G=\R;a\cdot b = a+b+1$.
    \begin{proof}
        We want to show that $G$ is a group.\vp
        \underline{(G1) Closure}\vp
        $G$ is closed because addition is closed under $\R$.\vpp
        \underline{(G2) Associativity}
        \begin{proof}
            We want to show that $a*(b*c)=(a*b)*c$.
            \begin{align*}
                (a*b)*c&=a+b+c+2\\
                a*(b*c)&=a+b+c+2\\
            \end{align*}
            Thus, $G$ is associative.
        \end{proof}
        \underline{(G3) Identity}
        \begin{proof}
            Let $e = -1$. We want to show that $a*e=e*a=a$
            \begin{align*}
                a*e = a + 1 - 1 = a\\
                e*a = -1 + a + 1 = a\\
            \end{align*}
            Hence, $G$'s identity is -1.
        \end{proof}
        \underline{(G4) Inverse}
        \begin{proof}
            Let $a\in G$. We want to show that $\exists a^{-1}\in G$ such that $a*a^-1 = e$. 
            \begin{align*}
                a + a^{-1} + 1&= -1\\
                a^{-1} &= -2 -a\in\R\\
            \end{align*}
            Hence, every element in $G$ has an inverse.
        \end{proof}
        Hence, $G$ is a group.
        \end{proof}
        \item \textit{Skip}
        \item $G=\{\epsilon,(1\;2),(1\;3),(1\;4) \}$.
        \begin{proof}
            We want to show that $G$ is a group.\vp
            \underline{(G1) Closure - FAILS}\vp
            $G$ is not closed:
            \begin{align*}
                (1\;2)(1\;3) = (1 3 2) \notin G
            \end{align*}
            \underline{(G2) Associativity - FAILS}
            $G$ fails under associativity:
            \begin{align*}
                (1\;2)(1\;3)\ne (1\;3)(1\;2)\\
                (1\;3\;2)\ne(1\;2\;3)
            \end{align*}
    \end{proof}
    \item $G=\{0,2,4,6\};\;(\Z_8,+)$.
    \begin{proof}
        We want to show that $G$ is a group.\vp
        \underline{(G1) Closure}\vp
        $G$ is closed under addition because the sum of two even numbers will always be even, and any even number over 8 will be reduces to an even number less than 8. Hence, $G$ is closed under addition $\Z_8$.\vpp
        \underline{(G2) Associativity}
        \begin{proof}
            We want to show that $a*(b*c)=(a*b)*c$.
            \begin{align*}
                (a*b)*c&=(a+b+c)\mod 8\\
                a*(b*c)&=(a+b+c)\mod 8\\
            \end{align*}
            Thus, $G$ is associative.
        \end{proof}
        \underline{(G3) Identity}
        \begin{proof}
            Let $e = 0$. We want to show that $a*e=e*a=a$, and $e\in G$. 
            \begin{align*}
                a*e = (a + 0) \mod 8= a\mod 8\\
                e*a = (0+a)\mod 8 = a \mod 8
            \end{align*}
            Hence, $G$'s identity is 0.
        \end{proof}
        \underline{(G4) Inverse}
        \begin{proof}
            Let $a\in G$. We want to show that $\exists a^{-1}\in G$ such that $a*a^{-1} = e$. 
            \begin{align*}
                a + a^{-1} &= 0\\
                a^{-1} &=\bar 8-a,\\
                \text{if }a \in \Z_8, &\text{ then consider that } a \text{ is even.}\\
                \text{Let } \bar 2 k = a. \text{ Thus, }&\bar 8 - a = \bar 2(\bar 4 - k)\in \{0,2,4,6\}.
            \end{align*}
            This holds because any sum of even numbers will result in another even number in mod 8.
        \end{proof}
        Hence, $G$ is a group.
    \end{proof}
    \item $G=\{16,12,8,4\}$
    \begin{proof}
        We want to show that $G$ is a group.\vp
        \underline{(G1) Closure}\vp
        $G$ is closed under multiplication, because any multiplication of two multiples of $4$ that are not divisible by $20$ will result in another multiple of 4 that is not divisible by 20.\vpp
        This is because every number can be represented as a multiple of four:
        \begin{enumerate}[itemsep=2pt, parsep=0pt, topsep=4pt, partopsep=0pt]
            \item $\overline {16} = 4\cdot \bar 4$
            \item $\overline {12} = 3 \cdot \bar 4$
            \item $\overline {8} = 2\cdot 4$
            \item $\bar 4 = 1\cdot \bar 4$
        \end{enumerate}
        However, because 1, 2, 3, \& 4 are all coprime with 5, no multiplication between any of those numbers will result in $5\cdot \bar 4 = \overline{20}$, which means that 
        the group is closed under the operation of multiplication in $\Z_{20}$.\newpage
        \underline{(G2) Associativity}
        Multiplication is associative by definition.
        \underline{(G3) Identity}
        \begin{proof}
            Let $e = 16$. Therefore, we have:
            \begin{enumerate}[itemsep=2pt, parsep=0pt, topsep=4pt, partopsep=0pt]
                \item $\bar {16} * \bar {16} = \bar {16}$
                \item $\bar {16} * \bar 12 = \bar 12$
                \item $\bar {16} * \bar 8 = \bar 8$
                \item $\bar {16} * \bar 4 = \bar 4$
            \end{enumerate}
        \end{proof}
        \underline{(G4) Inverse}
        \begin{proof}
            Every element also has an inverse such that $a \cdot a^{-1}=16$:
            \begin{enumerate}
                \item $\bar {16} * \bar {16} = \bar {16}$
                \item $\bar 8 * \bar 12 = \bar 16$
                \item $\bar 12 * \bar 8 = \bar 16$
                \item $\bar 4 * \bar 4 = \bar 16$
            \end{enumerate}
        \end{proof}
        Hence, $G$ is a group.
    \end{proof}
    \end{enumerate}
    
    \item Do Exercise 10 (a),(b). You should assume that $a$ and $b$ are different elements!

    \begin{enumerate}
        \item If $a^4=1$ and $ab=ba^2$ in a group, show that $a= 1$.
        \begin{proof}
            Because this a group closed under the operation $ab=ba^2$, we know that:
            \begin{align*}
                ab&=ba^2\\
                ab(a^2)&=ba^2(a^2)=b
            \end{align*}
            Substituting in $aba^2$ for $b$, we get:
            \begin{align*}
                ab&=ba^2\\
                a(aba^2)&=ba^2\\
                a^2ba^2\cdot a^2&=ba^2\cdot a^2\\
                a^2ba^4&=ba^4\\
                a^2&=1
            \end{align*}
        \end{proof}
        \item If $a^6=1$ and $ab=ba^3$ in a group, show that $a^2=1$ and $ab=ba$. 
        \begin{proof}
            Because this group is closed under $ab=ba^3$ and $a^6=1$, we know:
            \begin{align*}
                ab&=ba^3\\
                ab(a^3)&=b(a^6)=b
            \end{align*}
            Substituting in $aba^3$ for $b$, we get:
            \begin{align*}
                ab&=ba^3\\
                a(aba^3)&=ba^3\\
                a^2ba^3&=ba^3\\
                a^2ba^6&=ba^6\\
                a^2 &= 1
            \end{align*} 
            Then, because $a^2 = 1$, we know:
            \begin{align*}
                ab&=ba^3\\
                ab&=ba(a^2)\\
                ab&=ba
            \end{align*}
            Hence, $a^2=1$ and $ab=ba$. 
        \end{proof}
    \end{enumerate}
    \item Do Exercise 21. 
    \begin{enumerate}
        \item Show that a group $G$ is abelian if and only if $(gh)^2=g^2h^2$.
        \begin{proof}
            We want to show that $(gh)^2=g^2h^2$.
            $$(gh)^{-2}(gh)^2=g^2h^2(gh)^{-2}$$
            Thus,
            \begin{align*}
                e&=g^2h^2h^{-1}g^{-1}h^{-1}g^{-1}\\
                e&=g^2hg^{-1}h^{-1}g^{-1}\\
                g&=g^2h^1g^{-1}h^{-1}\\
                ghg&=g^2h^1\\
                g(hg)&=g(gh)\\
                hg&=gh
            \end{align*}
        \end{proof}
        \begin{proof}
            We want to show that if $G$ is abelian, $(gh)^2=g^2h^2$.
            \begin{align*}
                gh &= hg \\
                h &= g^-1hg\\
                e &= h^{-1}g^{-1}(hg)\\
                e &= (hg)^{-1}(hg)\\
                hg &= (hg)^{-1}(hg)^2\\
                hghg &= (hg)^2\\
                hhgg &= (hg)^2\\
                h^2g^2&= (hg)^2
            \end{align*}
        \end{proof}
    \end{enumerate}



    \item Do Exercise 28. This is a bit like Exercise 9 and you will need B\'{e}zout's Identity/GCD Theorem.
    \item Let $a$ and $b$ be elements of a group $G$. If $a^n = b^n$ and $a^m = b^m$ where $\gcd(m,n) = 1$, show that $a=b$. 
        \begin{proof}
            We want to show that $a = b$.\vp
            Because $\gcd(m,n)=1$, $m$ and $n$ are relatively prime, Let $x, y\in \Z$ such that $xm + yn = 1$. \vp
            Becuase $a^m = b^n$ and $a^m=b^m$, then, we write:
            \begin{align*}
                a^{xm}a^{yn}=a^{xm+yn}&=a^1\\
                (a^m)^x(a^n)^y&=a\\
                \text{Substitute in } b^n \;\&\; b^m &\text{ for } a^n \;\&\; a^m: \\
                (b^m)^x(b^n)^y= b^{mx}b^{my}&=b^1=a
            \end{align*}
            Hence, $a=b$. 
        \end{proof}
\end{enumerate}


\end{document}
%document class
\documentclass[10pt,oneside]{book}
%%%% Page Info + Commands %%%%%{

%packages
\usepackage{pgfplots}
\pgfplotsset{compat=1.18}
\usepackage{amssymb}
\usepackage{amsfonts}
\usepackage{amstext}
\usepackage{amsmath}
\usepackage{enumerate}
\usepackage{enumitem}
\usepackage{fancyhdr}
\usepackage{changepage}

%This will shrink the page
\newcommand{\smallpage}{  \setlength \oddsidemargin{.5in}
            \setlength \textwidth{5in}
            \setlength \topmargin{0in}
            \setlength \textheight{9in}}


% Commands for sets of numbers
\newcommand{\Z}{\mathbb{Z}}
\newcommand{\R}{\mathbb{R}}
\newcommand{\C}{\mathbb{C}}
\newcommand{\Q}{\mathbb{Q}}
\newcommand{\F}{\mathbb{F}}
\newcommand{\N}{\mathbb{N}}
\newcommand{\Ker}{{\rm Ker}\,}
\newcommand{\im}{{\rm Im}\,}

\newcommand{\n}{\vspace{0.15cm}\\}
\newcommand{\en}{\vspace{0.25cm}\\}

\newcommand{\gimage}[3]{
\begin{figure}[h]   
    \centering          
    \includegraphics[width=#2\textwidth]{#1}  
    \caption{#3}
    \label{fig:#1}
\end{figure}\\
}

\newcommand{\centerit}[1]{{\begin{center} #1 \end{center}}}
\newcommand{\ul}[1]{\underline{#1}}

%%%%%%%% command for graphics %%%%%%%%%%%%%
\usepackage{fancyhdr}
%}

%%%% Page Setup %%%%%%%
\smallpage\sloppy
\pagestyle{fancy}
\lhead{\large{\textbf{HW8 - Gordon Novak}}}
%\rfoot{\thepage/\pageref{LastPage} }
\setlength{\headheight}{10pt}
%\setcounter{section}{1}

\begin{document}

1. Consider the triple integral:
\[\int_{-4}^4\int_0^{\sqrt{16-x^2}}\int_{-\sqrt{x^2+y^2}}^0\sqrt{x^2+y^2+z^2}\,dz\,dy\,dx\]
Convert the integral from rectangular corodinates to cylindrical coordinates and spherical coordinates. 
\centerit{\ul{Cylindrical Coordinates}}
To convert to cylindrical coordinates, we just follow the cylindrical coordinate transformations:
\begin{align*}
    r^2 &= x^2 + y^2\\
    z &= z\\
    x &= r\cos\theta\\
    y &= r\sin\theta
\end{align*}
Now, we can start looking at these equations and simplifying. The integrand is rather straightforward:
\[\sqrt{x^2+y^2+z^2}=\sqrt{r^2+z^2}\]
When looking at the bounds, we can also do some more convenient conversions:
\[z=-\sqrt{x^2+y^2}\implies z=-\sqrt{r^2}=\pm r\]
Then, we can also look at the other bound:
\[y=\sqrt{16-x^2}\implies y^2+x^2=16\implies r = \pm 4\]
Additionally, we get that $y$ must start at 0, which is quite helpful, as this means that:
\[r\sin\theta = 0\]
And when combined with the $x$-bounds that say:
\[r\cos\theta = \pm r\]
We then know that with a max $r$ of $\pm 4$, that $cos\theta: \{-1, 1\}$ so $\theta: \{0, \pi\}$. (We don't go all the way to 2$\pi$, we'd need the $-\sqrt{16-x^2}$ for that.)
This then gives us a rather convenient integral:
\[\boxed{\int_0^{\pi}\int_0^4\int_{-r}^0r\sqrt{r^2+z^2}\;dz\,dr\,d\theta}\]
Note that the $r$ in the integral is our determinant of the jacobian for cylindrical coordinates.
\centerit{\ul{Spherical Coordinates}}
Essentially, we perform a very similar process here, just with spherical coordinates
\begin{align*}
    \rho^2&=x^2+y^2+z^2\\
    x&=\rho\sin\phi\cos\theta\\
    y&=\rho\sin\phi\sin\theta\\
    z&=\rho\cos\phi
\end{align*}
The integrand comes easily:
\[\sqrt{x^2+y^2+z^2}=\rho\]
Now, consider the $z$ bound in the original integral:
\[z=-\sqrt{x^2+y^2}\]
Solving gives us that
\[z=-\sqrt{\rho^2\sin^2\phi\cos^2\theta + \rho^2\sin^2\phi \cos^2\theta}=-\sqrt{\rho^2\sin^2\phi}=-\rho\sin\phi\]
Thus, because $z=\rho\cos\phi$, we get that:
\[\rho\cos\phi = -\rho\sin\phi\implies \phi = \tan^{-1}\left(-1\right)\]
We get bounds for $\phi:\{\frac{\pi}2, \frac{3\pi}4\}$ (Cause we also have that $\rho\cos\phi = 0$)\n
Hooray. Now we can start solving for $\rho$. We are given in the $y$-bounds for the original integral that:
\[y=\sqrt{16-x^2}\]
Which turns into:
\begin{align*}
    \rho\sin\phi\sin\theta &= \sqrt{16-\rho^2\sin^2\phi\cos^2\theta}\\
    \rho^2\sin^2\phi\sin^2\theta &= 16-\rho^2\sin^2\phi\cos^2\theta\\
    \rho^2\sin^2\phi(\cos^2\theta+\sin^2\theta)&=16\\
    \rho\sin\phi &= \pm 4
\end{align*}
And if we recall our bounds for $\phi$ earlier, we know that the max for $\phi$ will be $\sin\frac{\pi}2$. Thus, we get our bounds for $\rho$ as $\rho:\{0, \frac{4}{\sin\phi} \}$\n
Ok cool. Now time for theta. We are given that $x:\{-4, 4\}$. Thus, $\rho\sin\phi\cos\theta : \{-4, 4\}$.\n
This makes things easy as our max for $\rho$ was 4 and our horizontal for $\phi$ was $\frac{\pi}2$, So dividing out gives us that $\cos\theta: \{-1, 1\}$, so $\theta:\{0, \pi\}$.\n
Finally, after all that work we get the integral:
\[\boxed{\int_0^\pi\int_{\frac{\pi}2}^{\frac{3\pi}4}\int_0^\frac{4}{\sin\phi}\rho^3\sin\phi\;d\rho \,d\phi\,d\theta}\]
Note that I did put our jacobian in there.\\
\hrule \vspace{0.3cm}
2. Find the volume of the solid that lies inside $x^2+y^2+z^2=6$, outside $z=-\sqrt{x^2+y^2}$ and below the $xy$-plane. \n
Ok. We're going to use spherical coordinates to solve this problem because I see right out of the gate that:
\[x^2+y^2+z^2=6\implies \rho^2 = 6\]
Furthermore, "below the $xy$-plane" tells me that $\phi:\{\frac{\pi}2, \pi\}$. \n
Now here comes the tricky part. The equation $z=-\sqrt{x^2+y^2}$. If I do a quick conversion to cylindrical, I see that $z = -\sqrt{r^2}$, which simplifies to $z = -r$. Now converting to spherical, I get that:
\[\rho\cos\phi = -\rho\sin\phi\]
Now this is interesting, because instead of getting a new bound for $\rho$, I find that $\phi$ cannot go past $\tan^{-1}(-1)=\frac{\pi}4$. This means that our initially assumed bounds for $\phi$ must be limited to $\frac{3\pi}4$. (This is because an angle of $-\frac{\pi}4$ underneath the $z$-plane corresponds to $\phi = \frac{3\pi}4)$. \n
Now, we can evaluate our new-found integral for the volume:
{
\newcommand{\ia}{\int_0^{2\pi}}
\newcommand{\ib}{\int_{\frac{\pi}2}^{\frac{3\pi}4}}
\newcommand{\brk}[1]{\;\left[#1\right]\,}
\begin{align*}
   \ia\ib\int_0^{\sqrt{6}}1\cdot \rho^2\sin\phi\;\; d\rho \, d\phi\, d\theta&=\ia\ib \brk{\frac{1}3\rho^3\sin\phi\Big|_0^{\sqrt{6}}}\;\;d\phi d\theta\\
   &=\ia\ib 2\sqrt 6\sin\phi\;d\;\phi \,d\theta\\
   &=\ia\brk{-2\sqrt 6\cos\phi\Big|_{\frac{\pi}2}^{\frac{3\pi}4}}d\theta\\
   &=\ia 2\sqrt{3} \,d\theta = \boxed{4\pi\sqrt 3}
\end{align*}
Hooray done with that.\n
}
\hrule\vspace{0.3cm}
3. Set up an integral in spherical coordinates that calculates the mass of the solid above the cone $\phi=\frac{\pi}3$ and below the sphere $\rho=8\cos\phi$. The density of the solid is given by $f(x,y,z)=y^2$. \en
This question is rather straightforward. Our cone equation $\phi=\frac{\pi}3$ gives the bounds for $\phi$ as:
\[\phi:\{0,\frac{\pi}3\}\]
Next, we know our $\rho$ will extend from 0 to the sphere, which is simply given as $\rho=8\cos\theta$. Thus, our bounds for $\rho$ are:
\[\rho:\{0,8\cos\phi\}\]
Finally, we know that because all our bounds are symmetrical about the $z$-axis,
\[\theta:\{0,2\pi\}\]
Finally, for our equation $y^2$, converting to spherical coordinates gives:
\[f(x,y,z)=y^2\implies f(\rho,\theta,\phi)=\rho^2\sin^2\phi\sin^2\theta\]
Thus, adding the jacobian in to the integrand and pluggin in our bounds gives:
\[\boxed{\int_0^{2\pi}\int_0^{\frac{\pi}3}\int_0^{8\cos\phi}\rho^4\sin^3\phi\sin^2\theta\;\;d\rho\,d\phi\,d\theta}\]
\hrule\vspace{0.3cm}
4. Suppose you have an avocado in the shape of a sphere of radius 5. Suppose the pit of the avocado is a sphee of radius 2. Set-up a triple integral in spherical coordinates that calculates the volume of 3/4 of the avocado with the pit removed. \en
This question is also surprisingly straightforward relative to question one. First, we know via the question that:
\[\rho:\{2, 5\}\]
Furthermore, if we want only 3/4 of the avocado, we either restrict $\rho$ or $\theta$. I choose to restrict $\theta$ to only cover 3/4 of the avocado ($\theta:\{0,\frac{3\pi}2$). That gives the integral:
\[\boxed{\int_0^\frac{3\pi}4\int_0^\pi\int_2^51\cdot\rho^2\sin\phi\;\;d\rho\,d\phi\,d\theta}\]\newpage
\hrule\vspace{0.3cm}
5. Sketch the following vector fields. Be sure to sketch at least 8 vectors that show a good flow of the field. 
\begin{enumerate}[parsep=0pt]
    \item [(a)] $\mathbf F = -y^2\mathbf i + x\mathbf j$
    \item [(b)] $\mathbf F = (4y,x+2)$
    \item [(c)] $\mathrm F = \mathrm i + xy\mathrm j$
\end{enumerate}
For question \textbf{a}, all we have to do is consider that the x-coord is always negative, and increases in strength with y, and the y-coord is just dependent on x.\\
\gimage{image.png}{0.25}{Question a's vector plot}
Now for question 2, y can be negative and positive, and x's positivity and negativity will be offset by 2:
\gimage{image2.png}{0.25}{Question b's vector plot}
For question 3 it gets a little weird because $\mathrm i$ is just going to be 1, while $j$ will depend on both x and y, and will be positive in Q1 and Q3 and negative in Q2 and Q4 (because neg * neg = pos). 
\gimage{image3.png}{0.25}{Question c's vector plot}
Hooray!,


\end{document}
